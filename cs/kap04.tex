\chapter{Spinorový popis expandujících impulzních gravitačních vln}
\label{chap:kap04}
Expandující impulzní gravitační vlny jsou třída řešení Einsteinových rovnic odpovídající impulzní limitě
Robinson--Trautmanovy třídy řešení typu N. S běžným popisem expandujících gravitačních vln se lze seznámit např. v
\cite{Podolsky_1999} nebo \cite{Podolsky:2016mqg}, v této části se budeme zabývat
popisem expandujících impulzních vln v řeči $SL(2, \mathbb{C})$ spinorů.
%Tato část má sloužit zejména jako základní přehled výsledků spinorového popisu a představit možnosti rozšíření spinorového popisu
%vedoucí na reprodukci výsledků z prací, které používají jiný popis, a k výsledkům novým.

Spinorová pole jsou v obecné relativitě konstruována zejména pomocí tetrádového formalismu,
který je úzce spojen s Lorentzovskou invariancí tečných prostorů jednotlivých bodů v prostoročase.
V tomto formalismu vycházíme ze čtyř nulových vektorů -- nulové tetrády $(l^a, n^a, m^a, \bar{m}^a)$,
která je lokálně determinována \emph{spin--bází} $(o^A, \iota^A)$ podle relací
\begin{equation}
    \label{eq:tetrad-spinor-basis-relation}
    l^\mu = o^A o^{A'}, ~~ m^\mu = o^A \iota^{A'}, ~~ \bar{m}^\mu = \iota^A o^{A'}, ~~ n^\mu = \iota^A \iota^{A'},
\end{equation}
kde velké písmeno v indexu označuje spinorový index. Pro spinorové objekty zavádíme dva prostory,
jednomu z nich odpovídá nečárkovaný index, druhému z nich čárkované prostory. Operace komplexního sdružení
mapuje spinory z nečárkovaného spin--prostoru na čárkovaný a naopak.

Mezi spin--bázemi nečárkovaného a čárkovaného prostoru platí relace $\bar{o^A} = \bar{o}^{A'} = o^{A'}$, $\bar{\iota^A} = \bar{\iota}^{A'} = \iota^{A'}$ --
u spinorů z komplexně sdruženého prostoru explicitně nepíšeme operaci komplexního sdružení, čárkovaný index sám o sobě značí,
že se jedná o objekt v daném indexu komplexně sdružený ke svému nečárkovanému protějšku.

Formalismem dvoukomponentových spinorů v obecné teorii relativity se blíže zabývají publikace
\cite{penrose_rindler_1984}, \cite{doi:10.1142/5222}, případně původní práce Leopolda Infelda \cite{zbMATH03005509},
který se dvoukomponentovými spinory v obecné relativitě zabýval jako jeden z prvních. Na spinory se také běžně
nahlíží jako na prvky v lineárních reprezentacích Cliffordových algeber, v tomto smyslu se s nimi lze seznámit např. v \cite{fecko_2006}.

\section{Expandující impulzní gravitační vlny}
Pro popis expandujících gravitačních vln vycházíme zejména z  poznámek Scholtze \cite{scholtz_notes}, kde konstrukce prostoročasu následuje klasický článek
Penrose a Nutku \cite{PenroseNutku1992}, ve kterém představují možnost popisu impulzních gravitačních vln v řeči spinorů, a článek Alieva, Nutku a Yavuze \cite{Aliev2001},
ve kterém je řešení zobecněno na impulzivní limitu Robinon--Trautmanových řešení typu N.
Druhý zmíněný článek dále pokračuje zobecněním na řešení se zrychlením, kde vychází z ploché limity C-metriky, ve kterém konstruují impulzní vlnu metodou cut and paste.
V této části se budeme věnovat pouze tzv. nulovému případu, kdy je Robinson-Trautmanův parametr $k=0$.
Nejprve zavedeme Schwarzovu derivaci, která se ukáže být vhodná pro efektivní zápis spojité metriky expandujících impulzních vln,
dále se budeme věnovat samotnému popisu -- pomocí cut and paste metody vygenerujeme impulzní nadplochu a následně pomocí spinorového
formalismu přejdeme ke spojité metrice.
\subsection{Schwarzova derivace}
Nejobecnější bijektivní transformací $\mathbb{C} \to \mathbb{C}$ je Möbiova transformace
\begin{equation}
    \zeta \mapsto \frac{\alpha \zeta + \beta}{\gamma \zeta + \delta},
\end{equation}
která odpovídá projektivní grupě $PGL(2, \mathbb{C})$. Definujme Schwarzovu derivaci funkce $h(\zeta)$ jako
\begin{equation}
    \{h; \zeta\} = -\frac{1}{2}\left(\frac{h^{'''}}{h^{'}} - \frac{3}{2} \frac{h^{''2}}{h^{'2}}\right),
\end{equation}
tato konstrukce způsobuje anihilaci funkcí, které odpovídají Möbiově transformaci, tedy
\begin{equation}
    \left\{ \frac{\alpha \zeta + \beta}{\gamma \zeta + \delta}; \zeta \right\} = 0,
\end{equation}
a i naopak, jakákoliv funkce s nulovou Schwarzovou derivací je Möbiova transformace.

Dále je Schwarzova derivace vůči $PGL(2, \mathbb{C})$ invariantní, tedy
\begin{equation}
    \left\{ \frac{\alpha h(\zeta) + \beta}{\gamma h(\zeta) + \delta} ; \zeta \right\} = \left\{ h(\zeta); \zeta \right\}.
\end{equation}

Je vhodné zmínit, že pokud k Möbiově transformaci přidáme podmínku unimodularity (tedy $\alpha \delta - \beta \gamma = 1$),
bude se jednat o tzv. spin--transformace, které lze zapsat pomocí matic reprezentujícíh grupu $SL(2,\mathbb{C})$, která je dvojlistým nakrytím
omezené Lorentzovy grupy. Samotná unimodulární Möbiova transformace tvoří grupu $PSL(2, \mathbb{C})$.

\subsection{Cut and paste metoda}
Obdobně jako v případě neexpandujících impulzů začneme s metrikou Minkowského prostoročasu, ale s opačnou signaturou,
\begin{equation}
    \rmd s^2 = \rmd t^2 - \rmd x^2 - \rmd y^2 - \rmd z^2.
\end{equation}
Transofrmací \eqref{eq:retarded_coordinates} a \eqref{eq:complex_coordinates} přejdeme
k metrice
\begin{equation}
    \rmd s^2 = 2 \rmd \matu \rmd \matv - 2 \rmd \eta \rmd \bar{\eta}.
\end{equation}

Pro popis expandujících impulzů jsou vhodné souřadnice, které popisují nulový kužel s
vrcholem v kartézském počátku jako plochu konstantní souřadnice. Zavedeme tedy
souřadný systém $(v, u, \zeta, \bar{\zeta})$ jako
\begin{equation}
    \label{eq:expanding_null_tetrad}
    \matu = u,~~~ \matv = v + u \zeta \bar{\zeta}, ~~~ \eta = u \zeta, ~~~ \bar{\eta} = u \bar{\zeta},
\end{equation}
nulový kužel s vrcholem v kartézském počátku je pak dán jako plocha $v=0$. Metrika má v těchto souřadnicích tvar
\begin{equation}
    \label{eq:metric_stereographic_null}
    \rmd s^2 = 2 \rmd u ~ \rmd v - 2 u^2 \rmd \zeta ~ \rmd \bar{\zeta}
\end{equation}
a inverzní transformace ke kartézským souřadnicím jsou
\begin{align}
    t = \frac{1}{\sqrt{2}} \left(u + v + u \zeta \bar{\zeta}\right) & z = \frac{1}{\sqrt{2}} \left(v + u \zeta \bar{\zeta} - u\right) \\
    x = \frac{u}{\sqrt{2}} (\zeta + \bar{\zeta}) & y = \frac{u}{i \sqrt{2}} (\zeta - \bar{\zeta}).
\end{align}

Světelný kužel $v=0$ nazveme $N$, metrika na této nadploše je
\begin{equation}
    \label{eq:null_cone_metric}
    \rmd s_N^2 = -2 u^2 \rmd \zeta ~ \rmd \bar{\zeta}.
\end{equation}

Podél této nadplochy provedeme řez, obdržíme dvě části prostoročasu. Jako $M^+$ označíme vnitřek
svetelného kuželu, jako $M^-$ označíme zbytek prsotoročasu a souřadnice na $M^-$ budeme označovat jako $(\hat{u}, \hat{v}, \hat{\zeta}, \hat{\bar{\zeta}})$.
Tyto dvě části budeme chtít na nadploše $N$ zpětně slepit s \emph{warpem} - identifikací bodu $(v=0, u, \zeta, \bar{\zeta})$ na $N \subset M^+$
s bodem $(v = 0, \hat{u}, \hat{\zeta}, \hat{\bar{\zeta}})$, kde $\hat{\zeta}$ je holomorfní transformací $\zeta$,
\begin{equation}
    \label{eq:zeta_holomorfni_h}
    \hat{\zeta} = h(\zeta).
\end{equation}

Z požadavku spojitosti metriky na $N$ po slepení platí
\begin{equation}
    -2 u^2 \rmd \zeta ~ \rmd \bar{\zeta} = -2 \hat{u}^2 \rmd \hat{\zeta} ~ \rmd \hat{\bar{\zeta}}.
\end{equation}

Z toho plyne podmínka na transformaci souřadnice $u$,
\begin{equation}
    \label{eq:u_podminka_spojitosti}
    \hat{u} = \frac{u}{\left| h^{'}(\zeta) \right|},
\end{equation}
kde $h^{'}(\zeta) = h_{,\zeta}(\zeta)$ je derivace funkce $h(\zeta)$.
Obdrželi jsme lepící podmínky
\begin{equation}
    \label{eq:exp_penrose_lepici_podminky}
    (0, u, \zeta)_+ = \left(0, \frac{u}{\left| h^{'}(\zeta)\right|}, h(\zeta)\right)_{-}.
\end{equation}

\subsection{Odvození spojitého tvaru metriky pomocí spinorového formalismu}

Transformace \eqref{eq:exp_penrose_lepici_podminky} se odehrává na světelném kuželu
$v=0$, lze jí tedy chápat jako nelineární holomorfní transformaci spin--prostoru ve vrcholu
tohoto světelného kuželu \cite{PenroseNutku1992}.

Pro přechod ke spinorovému popisu použijeme Infeld-Van der Waerdenovy symboly, které pro kartézskou
vektorovou bázi odpovídají hermitovským maticím
\begin{equation}
    \label{eq:infeld-van-der-waerden}
    \begin{split}
        &\sigma_0^{AA'}=\frac{1}{\sqrt{2}}\left(\begin{matrix}
            1 & 0 \\
            0 & 1
        \end{matrix}\right)~,~~~ \sigma_1^{AA'}=\frac{1}{\sqrt{2}}\left(\begin{matrix}
            0 & 1 \\
            1 & 0
        \end{matrix}\right)~,\\
        &\sigma_2^{AA'}=\frac{1}{\sqrt{2}}\left(\begin{matrix}
            0 & -i \\
            i & 0
        \end{matrix}\right)~,~~~ \sigma_3^{AA'}=\frac{1}{\sqrt{2}}\left(\begin{matrix}
            1 & 0 \\
            0 & -1
        \end{matrix}\right),
    \end{split}
\end{equation}

spinorový ekvivalent "polohového"\ vektoru se složkami $x^\mu = (t, x, y, z)$ je
\begin{equation}
    \label{eq:spinor_vektor}
    x^{AA'} = \frac{1}{\sqrt{2}} \begin{pmatrix}
        t + z & x - iy \\
        x + iy & t - z
    \end{pmatrix} = \begin{pmatrix}
        \matv & \eta \\
        \bar{\eta} & \matu
    \end{pmatrix} = \begin{pmatrix}
        v + u \zeta \bar{\zeta} & u \zeta \\
        u \bar{\zeta} & u
    \end{pmatrix}.
\end{equation}

Spinor \eqref{eq:spinor_vektor} lze zapsat pomocí spinorovýh polí $\xi^A$ a $\mu^A$ jako
\begin{equation}
    \label{eq:spinor_position}
    x^{AA'} = u \xi^A \xi^{A'} + v \mu^A \mu^{A'},
\end{equation}
kde jednotlivé spinory mají složky
\begin{align}
    \label{eq:xi_mu_explicitne_v_bazi}
    \xi^A = \begin{pmatrix}
        \zeta \\ 1
    \end{pmatrix}, ~~~~~~ \mu^A = \begin{pmatrix}
        1 \\ 0
    \end{pmatrix}
\end{align}
ve spin--bázi $(o^A, \iota^A)$ svázané se souřadnicemi \eqref{eq:expanding_null_tetrad} podle \eqref{eq:tetrad-spinor-basis-relation}, tedy
přes příslušnou nulovou tetrádu.

Kovariantní zápis nulového vektoru $k^\mu$ pomocí spinorů je pak
\begin{equation}
    k^\mu = \xi^{A'} \sigma^\mu_{AA'} \xi^{A},
\end{equation}
kde $\sigma^\mu_{AA'}$ je soldering forma -- izomorfismus mezi prostory spinorů a tečným prostorem -- která
odpovídá Infeld-van der Waerdenovým symbolům \eqref{eq:infeld-van-der-waerden}.

Penroseovy lepící podmínky \eqref{eq:exp_penrose_lepici_podminky} v řeči 2-spinorů můžeme zapsat jako
identifikaci bodů na nulovém kuželu $x^\mu = u \xi^A \xi^{A'}$ a $\hat{x}^\mu = u \hat{\xi}^A \hat{\xi}^{A'}$,
kde z \eqref{eq:zeta_holomorfni_h} a \eqref{eq:u_podminka_spojitosti} platí
\begin{equation}
    \label{eq:xi_transformace}
    \hat{\xi}^A = \frac{1}{\sqrt{h'}} \begin{pmatrix}
        h \\ 1
    \end{pmatrix}.
\end{equation}

Pro stručnost píšeme $h(\zeta) = h$, $h'(\zeta) = h'$.

Metrika na $N$ lze zapsat jako
\begin{equation}
    \rmd s_N^2 = - 2 u^2 \left| \xi_A \rmd \xi^A \right|^2,
\end{equation}

kde ke snížení indexu u $\xi_A$ používáme Levi-Civitův spinor, který odpovídá
antisymetrické bilineární nedegenerované 2-formě na prostoru spinorů (a jeho komplexním sdružení),
\begin{equation}
    \xi_A = \varepsilon_{AB} \xi^B.
\end{equation}
Objekt $\xi_A$ je pak lineárním funkcionálem na prostoru spin--vektorů, považujeme ho za kovariantní spin--vektor.

Dále v souřadnicích \eqref{eq:expanding_null_tetrad} platí
\begin{equation}
    \xi_A \rmd \xi^A = \begin{pmatrix}
        -1 & \zeta
    \end{pmatrix}
    \begin{pmatrix}
        \rmd\zeta \\ 0
    \end{pmatrix} = -\rmd\zeta.
\end{equation}
Také pro transformovaný spinor $\hat{\xi^A}$ platí
\begin{equation}
    \hat{\xi}_A \rmd \hat{\xi}^A = \frac{1}{\sqrt{h'}} \begin{pmatrix}
        -1 & h
    \end{pmatrix}  \left( \rmd (h')^{-1/2} \begin{pmatrix}
        h \\ 1
    \end{pmatrix} + \frac{1}{\sqrt{h'}} \begin{pmatrix}
        h' \\ 0
    \end{pmatrix} \rmd \zeta \right) = -\rmd \zeta,
\end{equation}

forma $\xi_A \rmd \xi^A$ je invariantní vůči transformaci odpovídající
Penroseovým lepícím podmínkám. Invarianci této formy lze chápat jako
manifestaci spojitosti metriky procházející impulzní nadplochou $N$.

Dále ještě požadujeme, aby afinní parametr nulových generátorů $N$ nabýval stejné
hodnoty, nezávisle na tom jestli jej bereme z $M^+$ nebo $M^-$. Metrika celého
prostoročasu lze zapsat pomocí spinorů $\xi^A$ a $\mu^A$ pomocí relace \cite{doi:10.1142/5222}
\begin{equation}
    \varepsilon_{AB} \varepsilon_{A'B'} \sigma_{\mu}^{AA'} \sigma_{\nu}^{BB'} = g_{\mu \nu}
\end{equation}
jako
\begin{equation}
    \rmd s^2 = g_{\mu \nu} \rmd x^\mu \rmd x^\nu = \varepsilon_{AB} \varepsilon_{A'B'} \rmd x^{AA'} \rmd x^{BB'},
\end{equation}

kde po dosazení za $x^{AA'}$, resp. $x^{BB'}$ dostáváme
\begin{equation}
    \begin{split}
        \rmd s^2 &= 2 \rmd u ~ \rmd v - 2 u^2 ~ \left| \xi_A \rmd \xi^A \right|^2 + 4 \Re(\xi_A \rmd \mu^A)(u \rmd v - v \rmd u) \\
        &+ 4 u ~v ~ (\Re(\rmd \mu_A \rmd \xi^A) - \left|\mu_A \rmd \xi^A \right|^2) - 2 v^2 \left|\mu_A \rmd \mu^A \right|^2.
    \end{split}
\end{equation}

Jedná podmínka, kterou zatím na spinory pokládáme, je podmínka normalizace
\begin{equation}
    \mu_A \xi^A = 1.
\end{equation}
Spinory $\mu^A$ a $\xi^A$ tedy tvoří spin--bázi a pokud za ně explicitně dosadíme podle
\eqref{eq:xi_mu_explicitne_v_bazi}, metrika nabyde tvaru \eqref{eq:metric_stereographic_null}.

Položíme-li $v=0$, restringujeme se na impulzní nadplochu $N$ a metrika se redukuje na tvar
\begin{equation}
    \rmd s^2 = 2 \rmd u ~\rmd v - 2 u^2 ~\left| \xi^A \rmd \xi^A \right|^2 + 4 \Re(\xi_A \rmd \mu^A) ~u~ \rmd v.
\end{equation}

Z tohoto tvaru je třeba pro zajištění spojitosti metriky vyeliminovat člen s $\Re(\xi_A \rmd \mu^A)$. Oba spinory
$\xi^A, \mu^A$ jsou díky holomorfnosti funkce $h(\zeta)$, udávající\emph{warp}, také holomorfní. Aby tedy reálná část
funkce $\xi_A \rmd \mu^A$ byla nulová, musí být nulová celá funkce. Uvidíme, že triviální možnost konstantní funkce i přes svou holomorfnost
nebude vyhovovat, Schwarzova derivace konstanty je 0 a nepředstavuje \emph{warp} generující impulzní vlnu.

Splnění podmínek 
\begin{equation}
    \label{eq:podminky_na_kontinuitu}
    \xi_A \rmd \xi^A = \text{invariant}, ~~~~ \mu_A \xi^A = 1,~~~~ \xi_A \rmd \mu^A = 0
\end{equation}
nám zajišťuje spojitost celé metriky při průchodu impulzní nadplochou.

Už víme, že spinor $\xi^A$ se průchodem přes impulzní plochu transformuje jako $\xi^A \mapsto \frac{1}{\sqrt{h'}} \begin{pmatrix}
    h \\ 1
\end{pmatrix}$. Dále potřebujeme transformaci spinoru $\mu^A$ ve tvaru, ve kterém splňuje \eqref{eq:podminky_na_kontinuitu}.
Takový tvar je odvozen v \cite{scholtz_notes}, případně v \cite{Aliev2001} (s normalizací na $-1$),
\begin{equation}
    \label{eq:mu_transformace}
    \hat{\mu^A} = \sqrt{h'} \begin{pmatrix}
        1 - \frac{h ~h''}{2 h'^2} \\ ~ \\ - \frac{h''}{2 h'^2}
    \end{pmatrix}
\end{equation}

Dosazením těchto výsledků do \eqref{eq:spinor_vektor} za využití \eqref{eq:spinor_position} obdržíme na $M^+$
transformaci
\begin{equation}
    \label{eq:refraction_spinor}
    \begin{split}
        \matv &= u \frac{\left|h\right|^2}{\left|h'\right|} + v \left| h' \right| \left| 1 - \frac{h ~ h''}{2 h'} \right|^2 \\
        \matu &= \frac{u}{\left|h'\right|} + \frac{v}{4} \left| \frac{h''^2}{h'^3}\right| \\
        \eta &= \frac{u ~ h}{ \left| h' \right|} - v ~ \left| h' \right| \left(1 - \frac{h ~ h''}{2 h'^2} \right) \frac{\bar{h}''}{2 \bar{h}'^2}.
    \end{split}
\end{equation}

Ve spinorovém zápisu má metrika, splňující podmínky \eqref{eq:podminky_na_kontinuitu} tvar
\begin{equation}
    \rmd s^2 = 2 \rmd u ~ \rmd v - 2 u^2 \left| \xi_A \rmd \xi^A \right|^2 + 4 u~v~ \Re(\rmd \mu_A \rmd \xi^A) - 2 v^2 \left|\mu_A \rmd \mu^A\right|^2.
\end{equation}

Diferenciály spinorů lze z \eqref{eq:xi_transformace} a \eqref{eq:mu_transformace} vyjádřit pomocí Schwarzovy derivace jako
\begin{equation}
    \rmd \hat{\xi}^A = \hat{\mu}^A \rmd \zeta, ~~~~~~ \rmd \hat{\mu}^A = \{h; \zeta\} \hat{\xi}^A \rmd \zeta,
\end{equation}
na $M^+$ má metrika tvar
\begin{equation}
    \rmd s_+^2 = 2 \rmd u ~ \rmd v - 2 \left| u ~ \rmd \bar{\zeta} + v \{h; \zeta \} \rmd \zeta \right|^2
\end{equation}
a na $M^-$ tvar
\begin{equation}
    \rmd s_+^2 = 2 \rmd u ~ \rmd v - 2 u^2 ~ \rmd \zeta ~ \rmd \hat{\zeta}
\end{equation}

S použitím Heavisideovy theta funkce můžeme zapsat kompletní spojitou metriku
jako
\begin{equation}
    \rmd s^2 = 2 \rmd u ~ \rmd v - 2 \left| u ~ \rmd \bar{\zeta} + v ~ \Theta(v) \{h;\zeta\} ~ \rmd \zeta \right|^2.
\end{equation}

Z tohoto tvaru je pak zřejmé, že funkce $h$ ve tvaru Möbiovy transformace negenerují impulzní vlnu,
Schwarzova derivace je anihiluje a metrika v takovém případě přechází na plochý tvar.

\subsection{Refrakční rovnice pro geodetický pohyb}
Jelikož jsou souřadnice $u, v, \zeta$ globálně definované a spojité, můžeme je využít pro odvození refrakčních rovnic
geodetického pohybu při průchodu impulzní nadplochou $N$ odpovídající $v = 0$.
Porovnáním \eqref{eq:refraction_spinor} s  \eqref{eq:expanding_null_tetrad} za podmínky $v=0$ dostáváme sadu refrakčních rovnic pro
polohy v souřadnicích pozadí $\matu, \matv, \eta$
\begin{align}
    \matv^+_\rmi &= \left| \frac{h_\rmi}{\zeta_\rmi}\right|^2 \frac{\matv^-_\rmi}{\abs{h_\rmi'}}, & \matu^+_\rmi &= \frac{\matu^-_\rmi}{\abs{h_\rmi'}}, & \eta^+_\rmi &= \frac{h_\rmi}{\zeta_\rmi} \frac{\eta^-_\rmi}{\abs{h'_\rmi}}.
\end{align}
Používáme zde obdobné značení jako v kapitolách \ref{chap:kap02} a \ref{chap:kap03},
veličina s indexem $\rmi$ je vyčíslená na impulzní nadploše $N$, $h_\rmi = h(\zeta_\rmi)$ a horní index $\pm$ znamená příslušnost k souřadnicím na $M^\pm$.

Pro sadu refrakčních rovnic pro čtyřrychlost je třeba provést derivaci spojitých souřadnic podle afinního parametru a provést jednostranné limity $v \to 0$.
Jejich porovnáním obdržíme sadu rovnic lineárních ve složkách rychlostí,
\begin{equation}
    \begin{split}
        \dot{\matv}_\rmi^+ &= a_\matv \dot{\matv}_\rmi^- + b_\matv \dot{\matu}_\rmi^- + c_\matv \dot{\eta}_\rmi^- + \bar{c}_\matv \dot{\bar{\eta}}_\rmi^-, \\
        \dot{\matu}_\rmi^+ &= a_\matu \dot{\matv}_\rmi^- + b_\matu \dot{\matu}_\rmi^- + c_\matu \dot{\eta}_\rmi^- + \bar{c}_\matu \dot{\bar{\eta}}_\rmi^-, \\
        \dot{\eta}_\rmi^+ &= a_\eta \dot{\matv}_\rmi^- + b_\eta \dot{\matu}_\rmi^- + c_\eta \dot{\eta}_\rmi^- + \bar{c}_\eta \dot{\bar{\eta}}_\rmi^-,
    \end{split}
\end{equation}
koeficienty $a_i, b_i, c_i, d_i$ pro stručnost neuvádíme. Tento výpočet explicitně provedli Podolský, Sämann, Steinbauer a Švarc v
\cite{Podolsky:2016mqg} pro obecnější třídu prostoročasů, kde také ukazují, že $C^1$-matching je ve smyslu Filippovových řešení (viz. kapitola \ref{chap:kap02})
matematicky korektní technika konstrukce geodetik i v prostoročasech s expandujícími vlnami.
Zajímavou možností pro budoucí studie je odvození refrakčních rovnic pomocí spinorového formalismu.


\textcolor{red}{Zde asi konec, nepříjde mi, že by další sekce přinesla větší náhled, jen může zformalizovat proč klademe
příslušné podmínky na spinory při odvození}
\subsection{Geometrie impulzní nadplochy}

Impulzní nadplocha $N$ je nulová plocha rozdělující celý prostoročas na dva regiony, $M^-$ a $M^+$.
Tyto regiony jsou slepeny zpět s \emph{warpem} -- jsou identifikovány odlišné body a na $N$ dochází k
nespojitosti. Penrose zavádí tři typy geometrie na $N$.
První typ geometrie (Penroseův typ I) zajišťuje existenci takové metriky na $N$, že metriky indukované z $M^+$ a $M^-$
se s tou na $N$ shodují.

Penroseův II. typ geometrie navíc předpokládá, že pro všechny nulové generátory $\gamma$ nadplochy $N$
je afinní parametr stejný, nehledě na to, jestli $N$ uvažujeme jako hranici $M^+$ nebo $M^-$.
V geometrii II. typu lze zadefinovat paralelní transport vektorů tečných ke $\gamma$.

Penroseův III. typ geometrie pak umožňuje paralelní transport libovolného vektoru z libovolného tečného prostoru
na $N$ podél generátorů $\gamma$. V tomto typu geometrie tedy lze paralelně přenášet celá vektorová pole.

Nejprve se zaměříme na nejslabší geometrii I. typu. Mějme vektor $l^\mu$ tečný ke generátoru
nulové nadplochy $N$,
\begin{equation}
    l^\mu = o^A o^{A'},
\end{equation}
kde předpokládáme hladkost $o^A$ na $N$.

Ve volbě spinoru $o^A$ máme kalibrační volnost $o^A \mapsto \lambda o^A$, kde $\lambda$ je libovolná komplexní funkce.
Objekt, který se transformuje jako
\begin{equation}
    \eta \mapsto \lambda^p \bar{\lambda}^q \eta
\end{equation}
nazveme objektem typu $(p, q)$. V geometrii I. typu můžeme hladce roznést $o^A$ do
$M^+$ i $M^-$, obecně ale $o^A$ nebude hladký na $M^+ \cup M^-$. Na $M^+$ a $M^-$ máme
metrické a symetrické (ve smyslu afinní konexe) kovariantní derivace $\stackrel{+}{\nabla}_\mu$ a $\stackrel{-}{\nabla}_\mu$. Symbolem
$\stackrel{\pm}{\nabla}_\mu$ je myšlena kovariantní derivace na příslušeném prostoru $M^+$ nebo $M^-$.
Derivace ve směru tečném na $N$ je pak
\begin{equation}
    k^\mu \stackrel{\pm}{\nabla}_\mu,
\end{equation}
kde $k^\mu = \xi^A o^{A'} + \xi^{A'} o^A$ je libovolný vektor tečný k $N$.

Kovariantní derivací můžeme přes příslušnou soldering formu působit na spinory,\textcolor{red}{Zeptat se na GM II}
všechny derivace ve směru tečném k $N$ jsou lineárními kombinacemi
\begin{equation}
    o^A \stackrel{\pm}{\nabla}_{AA'}, ~~~ o^{A'} \stackrel{\pm}{\nabla}_{AA'}.
\end{equation}

Afinní konexe (a tedy kovariantní derivace) na $M^+$ a $M^-$ lze sjednotit do jedné kovariantní derivace
\begin{equation}
    \nabla_\mu = \Theta(v) \stackrel{+}{\nabla}_\mu + (1-\Theta(v)) \stackrel{-}{\nabla}_\mu.
\end{equation}

Jelikož je $\Theta$ konstantní na $N$, platí $l^a \nabla_a \Theta = 0$ a můžeme 
