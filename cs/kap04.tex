\newcommand{\Ab}{{\boldsymbol{A}}}
\newcommand{\Bb}{{\boldsymbol{B}}}
\newcommand{\Cb}{{\boldsymbol{C}}}
\newcommand{\Db}{{\boldsymbol{D}}}

\chapter{Spinorový popis expandujících gravitačních vln}
V této kapitole zavedeme spinorový formalismus na Minkowského prostoročase, následně jej použijeme k popisu expandujících
impulsních gravitačních vln. S běžným popisem expandujících gravitačních vln se lze seznámit např. v
\cite{Podolsky_1999}, \cite{Podolsky:2016mqg}, \textcolor{red}{další...}.
Spinorový formalismus vybudujeme na základě geometrického přístupu i přes široce uznávaný názor, že
nejpřirozenější popis spinorů je v řeči teorie reprezentací (např. \cite{fecko_2006}). Geometrická konstrukce spinorů je ale pro použití v relativitě jakožto geometrické teorii,
příhodnější a tohoto přístupu pak využívají právě práce zabývající se čistě relativistickým využitím spinorů, například
\cite{penrose_rindler_1984}, \cite{doi:10.1142/5222}.
\section{Spinorový formalismus}
\subsection{Krátký úvod do 2-spinorů}
Definujme zde Minkowského prostor $\mathbb{M}$ jako čtyřdimenzionální vektorový prostor nad $\mathbb{R}$ s Lorentzovskou metrikou
$\eta_{\mu \nu} = diag(1, -1, -1, -1)$. V každém bodě prostoru $\mathbb{R}^4$ existuje množina bázových vektorů z $\mathbb{M}$ kterou
nazýváme tetrádou.
Vezměme světelný kužel v $\mathbb{M}$, tedy úplný podprostor, kde pro každý vektor platí
\begin{equation}
    \eta_{\mu \nu} x^\mu x^\nu = 0.
\end{equation}

Každému časupodobnému nebo světelnému vektoru v prostoročasu přiřazujeme orientaci vůči počátku světelného
kužele, může mířit do minulosti nebo do budoucnosti, světelný kužel tak rozdělíme na budoucí a minulý.
Průnikem svetelného kuželu s plochou konstantního času dostaneme sféru v $\mathbb{R}^3$. Pokud leží nadplocha
konstantního času v souřadnici $x^0 = t =1$, představuje vzniklá sféra tzv. Riemannovu sféru $\Sigma$ pro kterou
platí rovnice
\begin{equation}
    x^2 + y^2 + z^2 = 1
\end{equation} 

Stereografickou projekcí pak můžeme ztotožnit sféru $\Sigma$ s rozšířenou komplexní rovinou $\tilde{\mathbb{C}}$
(tedy $\mathbb{C} \cup \{\infty\}$) a body $x$, $y$ a $z$ na sféře můžeme popsat jednou komplexní souřadnicí,
které se obvykle říká stereografická souřadnice,
\begin{equation}
    \xi = \frac{x + iy}{1-z},
\end{equation}
případně v polárních souřadnicích $(\theta, \phi)$
\begin{equation}
    \xi = e^{i \phi} \cos \frac{\theta}{2}.
\end{equation}
Inverzní vztahy jsou pak
\begin{equation}
    x=\frac{\xi+\bar{\xi}}{\xi \bar{\xi}+1},~~~y=\frac{\bar{\xi}-\xi}{\xi \bar{\xi}+1},~~~z=\frac{\xi\bar{\xi}-1}{\xi\bar{\xi}+1}.
\end{equation}

Dále zavedeme složky spinoru $\xi^\Ab$, $\xi^A = (\zeta, \eta)$, jako
\begin{equation}
    \label{eq:zavedeni_slozek_spinoru_z_stereograf_souradnice}
    \xi = \frac{\zeta}{\eta},
\end{equation}
vyhneme se tak nekonečné hodnotě stereografické souřadnice pro horní (severní) pól Riemannovy sféry. Nyní
potřebujeme udělat jistou formalizaci indexové notace.
Pokud je horní (a později i dolní) index tučný, jedná se o tzv. abstraktní index, který není svázaný se
spinorovou bází, kterou zavedeme dále, ale pouze naznačuje strukturu objektu. Pokud index není tučný, jedná
se přímo o souřadnicový index, který nabývá hodnot $1, 2$ a je svázaný s bází. Je zřejmé, že spinor $\xi^\Ab$
a stereografická souřadnice $\xi$ jsou i přes použití podobného značení
dva odlišné objekty.
Pomocí komponent spinoru lze popsat libovolný bod $(t, x, y, z)$ na světelném kuželu jako
\begin{equation}
    \label{eq:inverse_spinor_souradnicove}
    \begin{split}
        t &= \frac{1}{\sqrt{2}} \left( \zeta\bar{\zeta} + \eta\bar{\eta} \right) \\
        x &= \frac{1}{\sqrt{2}} \left( \zeta\bar{\eta} + \eta\bar{\zeta} \right) \\
        y &= \frac{1}{\sqrt{2}} \left( \bar{\zeta}\eta - \eta\bar{\zeta} \right) \\
        z &= \frac{1}{\sqrt{2}} \left( \zeta\bar{\zeta} - \eta\bar{\eta} \right).
    \end{split}
\end{equation}
Na prostoru spinorů se dále zavádí 2-forma $[ \cdot , \cdot ]$, která je
\begin{enumerate}[(1)]
    \item antisymetrická,
    \item bilineární a
    \item nedegenerovaná.
\end{enumerate}
Všimněme si, že kombinací prvních dvou podmínek dostaneme pro dva lineárně závislé spinory
(tedy pro dvojici $(\xi^\Ab, \kappa^\Ab)$ takové, že $\kappa^\Ab = \lambda \xi^\Ab$,
$\lambda \in \mathbb{C}$)
\begin{equation}
    [\xi^\Ab, \kappa^\Ab] = 0.
\end{equation}
Z třetí podmínky ale máme
\begin{equation}
    [\xi^\Ab, \eta^\Ab] \neq 0
\end{equation}
pro všechny $\xi^\Ab$, $\eta^\Ab$ z prostoru spinorů.

Tuto 2-formu obvykle zapisujeme symbolem $\epsilon_{\Ab \Bb}$, jedná se o tzv. Levi-Civitův spinor a na prostoru
spinorů zavádí skalární součin
\begin{equation}
    \label{eq:pusobeni_levicivitova_symbolu_na_spinory1}
    \epsilon_{\Ab \Bb} \xi^\Ab \eta^\Bb = \xi_\Bb \eta^\Bb = -\xi^\Ab\eta_\Ab
\end{equation}
a zastává tedy funkci obdobnou metrickému tenzoru. Zápis \eqref{eq:pusobeni_levicivitova_symbolu_na_spinory1} lze
také vyjádřit jako
\begin{equation}
    \label{eq:pusobeni_levicivitova_symbolu_na_spinory2}
    \xi^\Ab = \epsilon^{\Ab\Bb} \xi_\Bb, ~~~~ \xi_\Ab = \epsilon_{\Bb\Ab} \xi^\Bb,
\end{equation}
kde platí
\begin{equation}
    \epsilon^{\Ab\Cb} \epsilon_{\Bb\Cb} = \delta_\Bb^\Ab.
\end{equation}
Dále se zavádí spin-báze $(o^\Ab, \iota^\Ab)$ tak, aby platilo
\begin{equation}
    \label{eq:noramlizacni_podminky_spinorove_baze}
    [o^\Ab, \iota^\Ab] = -[\iota^\Ab, o^\Ab] = 1.
\end{equation}

Tak formálně propojíme souřadnicový zápis $\xi^A$ se samotným spinorem $\xi^\Ab$
\begin{equation}
    \xi^\Ab = \xi^0 o^\Ab + \xi^1 \iota^\Ab
\end{equation}

Z prvků báze lze pomocí dyadického součinu sestrojit Levi-Civitův spinor. S využitím normalizačních podmínek
\eqref{eq:noramlizacni_podminky_spinorove_baze} a vlastnosti \eqref{eq:pusobeni_levicivitova_symbolu_na_spinory2}
dostaneme vztah
\begin{equation}
    \epsilon_{\Ab\Bb} = o_\Ab \iota_\Bb - o_\Bb \iota_\Ab.
\end{equation}

V transformaci spinorových složek na reálné souřadnice \eqref{eq:inverse_spinor_souradnicove} máme komplexně
sdružené složky spinorů, operace komplexního sdružení ale není uzavřená na prostor spinorů, což lze jednoduše
ukázat když sečteme složky spinoru se složkami komplexně sdruženého - výsledný objekt má reálné složky.
Operace komplexního sdružení tedy zobrazuje spinory na prostor komplexně sdružených spinorů a píšeme
\begin{equation}
    \overline{\xi^\Ab} = \bar{\xi}^{\Ab'}.
\end{equation}
Komplexním sdružením spinorové báze dostáváme bázi prostoru komplexně sdružených spinorů
\begin{equation}
    \overline{o^\Ab} = \bar{o}^{\Ab'} = o^{\Ab'}, ~~ \overline{\iota^{\Ab}} = \bar{\iota}^{\Ab'} = \iota^{\Ab'}.
\end{equation}
U Levi-Civitova spinoru je stejně jako u složek spin-báze konvence psát $\overline{\epsilon_{\Ab\Bb}} = \epsilon_{\Ab'\Bb'}$ a ne 
$\bar{\epsilon}_{\Ab'\Bb'}$ jak bychom mohli očekávat.

Nejobecnějším typem spinoru je pak spinor s tzv. valencí $(p, q; r, s)$ který zapisujeme jako
\begin{equation}
    \chi^{\Ab\dots \Cb ~\boldsymbol{S}'\dots \boldsymbol{U}'}_{~~~~~~~~~~~\Db\dots \boldsymbol{F}~ \boldsymbol{W}'\dots \boldsymbol{Y}'},
\end{equation}

který má $p$ kontravariantních nečárkovaných indexů, $q$ kontravariantních čárkovaných indexů, $r$
kovariantních nečárkovaných indexů a $s$ kovariantních čarkovaných indexů. Obecně platí
\begin{equation}
    \chi^{\Ab\Bb'}_{~~~~\Cb\Db'} = \chi^{\Bb'\Ab}_{~~~~\Db'\Cb} = \chi^{\Ab~~\Bb'}_{~~\Cb~~\Db'},
\end{equation}
tedy čárkované indexy můžeme libovolně prohazovat s něčárkovanými, musí ale zůstat pozice čárkovaných mezi sebou
a nečárkovaných mezi sebou.

Nyní jsme kompletně vybaveni k přepsání vztahů \eqref{eq:inverse_spinor_souradnicove} do zápisu čistě pomocí spinorů.
Využijeme k tomu Infield-van der Waerdenovy symboly $\sigma_\mu^{\Ab\Ab'}$. Ty můžeme v Minkowského prostoru reprezentovat
čtyřmi hermitovskými maticemi
\begin{equation}
    \begin{split}
        &\sigma_0^{AA'}=\frac{1}{\sqrt{2}}\left(\begin{matrix}
            1 & 0 \\
            0 & 1
        \end{matrix}\right)~,~~~ \sigma_1^{AA'}=\frac{1}{\sqrt{2}}\left(\begin{matrix}
            0 & 1 \\
            1 & 0
        \end{matrix}\right)~,\\
        &\sigma_2^{AA'}=\frac{1}{\sqrt{2}}\left(\begin{matrix}
            0 & -i \\
            i & 0
        \end{matrix}\right)~,~~~ \sigma_3^{AA'}=\frac{1}{\sqrt{2}}\left(\begin{matrix}
            1 & 0 \\
            0 & -1
        \end{matrix}\right).
    \end{split}
\end{equation}

Tyto symboly nám dávají propojení mezi \textcolor{red}{world-tensory (překlad)} a spinory. Komponenty
\eqref{eq:inverse_spinor_souradnicove} vektoru ležícího na světelném kuželu pomocí spinoru se
složkami $\xi^A=(\zeta, \eta)$ zapíšeme jako
\begin{equation}
    x^\mu = \bar{\xi}^{A'} \sigma_{AA'}^\mu \xi^A.
\end{equation}

Dále zavedeme spinorovou formu vektoru $x^\mu$
\begin{equation}
    x^{AA'} = \sigma_\mu^{AA'} x^\mu = \frac{1}{\sqrt{2}} \begin{pmatrix}
        t + z & x - iy \\
        x + iy & t - z
    \end{pmatrix}.
\end{equation}
\textcolor{red}{Je tohle správně? Je to jen speciální případ na Minkowského PČ - spinová
struktura na varietě $\mathcal{M}$ je ekvivariantní lift frame bandlu - rozmyslet jak ten
funguje}
Z Levi-Civitových spinorů a Infield-van der Waerdenových symbolů můžeme zkonstruovat metriku
\begin{equation}
    g_{\mu \nu} = \epsilon_{AB} \epsilon_{A'B'} \sigma_\mu^{AA'} \sigma_\nu^{BB'}.
\end{equation}
Obecně můžeme s pomocí těchto symbolů vyjádřit spinory libovolné valence jako tenzory a
naopak.

Ve výpočtech se obvykle Infield-van der Waerdenovy symboly vynechávají a rovnou se píše
\begin{equation}
    g_{\mu \nu} = \epsilon_{AB}\epsilon_{A'B'}.
\end{equation}
Z tvaru Infield-van der Waerdenových symbolů je vidět korespondence mezi tenzorovými a spinorovými indexy,
každému tenzorovému indexu odpovídají dva spinorové.

Pomocí spinorové báze můžeme zavést světelnou (nulovou) tetrádu vztahy
\begin{equation}
    \begin{split}
        l^a &= o^A o^{A'}, \\
        n^a &= \iota^A \iota^{A'}, \\
        m^a &= o^A \iota^{A'}, \\
        \bar{m}^a &= \iota^A o^{A'}
    \end{split}
\end{equation}
kde $l^a, n^a, m^a$ a $\bar{m}^a$ jsou normalizované světelné vektory tvořící bázi.



\section{Expandující gravitační vlny}




\subsection{Refrakční rovnice}