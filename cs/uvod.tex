\chapter*{Úvod}
\addcontentsline{toc}{chapter}{Úvod}

\section*{Obecná teorie relativity}

Roku 1915 publikoval Albert Einstein Obecnou teorii relativity, nejmodernější a doposud nejúspěšnější teorii gravitace. Za více jak 100 let od její formulace byly v nesčetném množství
experimentů úspěšně ověřeny její předpovědi, včetně relativně nedávné první detekce gravitačních vln na interferometrech LIGO v Livingstonu a Hanfordu, která byla následována
desítkami dalších detekcí k nimž nyní přispívá i evropský projekt VIRGO. \textcolor{red}{Další zajímavé experimentální ověření?}.

Obecná teorie relativity je geometrická teorie popisující chování hmoty a energie v prostoročasu, který reprezentuje jako diferenciální varietu vybavenou metrickám tenzorem $g_{\mu \nu}$.
Tvar metrického tenzoru, resp. jeho složek (metrických funkcí) je s fyzikální realitou gravitačního pole spojen Einsteinovými polními rovnicemi, které mají v geometrizovaných jednotkách
tvar
\begin{equation}
    \label{eq:einsten_field_equations}
    R_{\mu \nu} - \frac{1}{2} R g_{\mu \nu} + \Lambda g_{\mu \nu} = 8 \pi T_{\mu \nu}.
\end{equation}
Levá strana polních rovnic představuje geometrii na diferenciální varietě reprezentující prostoročas, $R_{\mu \nu}$ je Ricciho tenzor křivosti, $R$ je Ricciho skalární křivost a $\Lambda$
představuje tzv. kosmologickou konstantu. Tenzor energie a hybnosti $T_{\mu \nu}$ na pravé straně pak dává spojení geometrických objektů s fyzikálním modelem, představuje rozložení hmoty,
energie a jejich toky a hybnosti. Jedná se o 10 nelineárních parciálních diferenciálních rovnic druhého řádu, které 