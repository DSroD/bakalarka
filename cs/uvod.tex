\chapter*{Úvod}
\addcontentsline{toc}{chapter}{Úvod}

\section*{Obecná teorie relativity}

Roku 1915 Albert Einstein představil a publikoval Obecnou teorii relativity \cite{Einstein1915}, toho času nejmodernější a doposud nejúspěšnější teorii gravitace. Za více jak 100 let od její formulace byly v nesčetném množství
experimentů úspěšně ověřeny její předpovědi. První dva klíčové úspěchy této teorie byly objasnění anomálního stáčení perihelia Merkuru následované pozorováním ohybu paprsků světla procházejících v blízkosti Slunce
při jeho zatmění sirem Eddingtonem v roce 1919.
Technicky nejsložitější a nejdéle očekávaným testem obecné relativity byla první detekce gravitačních vln v roce 2015 na interferometrech LIGO ve spojených státech, konkrétně v Livingstonu a Hanfordu \cite{PhysRevLett.116.061102}, která byla následována
desítkami dalších detekcí k nimž nyní přispívá i evropský projekt VIRGO.


Formálně je obecná teorie relativity geometrická teorie gravitace popisující chování hmoty a energie ve čtyřrozměrném prostoročase, který reprezentuje jako diferencovatelnou varietu $\mathcal{M}$ vybavenou metrickým tenzorem $g_{\mu \nu}$.
Tvar metrického tenzoru, resp. jeho složek (metrických funkcí) je s fyzikální realitou gravitačního pole spojen Einsteinovými rovnicemi, které mají v přirozených geometrizovaných jednotkách
tvar
\begin{equation}
    \label{eq:einsten_field_equations}
    R_{\mu \nu} - \frac{1}{2} R g_{\mu \nu} + \Lambda g_{\mu \nu} = 8 \pi T_{\mu \nu},
\end{equation}
přičemž uvažujeme $c = G = 1$.
Technicky se jedná o 10 nelineárních parciálních diferenciálních rovnic druhého řádu, jejichž levá strana představuje geometrii na varietě $\mathcal{M}$ reprezentující prostoročas, $R_{\mu \nu}$ je Ricciho tenzor křivosti, $R$ je Ricciho skalární křivost a $\Lambda$
je takzvaná kosmologická konstanta. Tenzor energie a hybnosti $T_{\mu \nu}$ na pravé straně pak dává spojení geometrických objektů s fyzikálním modelem, představuje rozložení hmoty,
energie a jejich toky a hybnosti. Tyto rovnice tedy říkají, jak hmota ovlivňuje geometrii prostoročasu a jak geometrie prostoročasu udává dynamiku jeho fyzikálního obsahu.
Obecně je řešení této soustavy velmi obtížný problém, získání přesných řešení vyžaduje řadu zjednodušujících předpokladů, jako například
předpoklady symetrie, speciální algebraické vlastnosti tenzoru křivosti, či specifické chování kongruencí testovacích částic, viz přehledové monografie \cite{griffiths_podolsky_2009}, \cite{stephani2003}. Běžně se k řešení těchto rovnic využívá poruchových metod a metod
numerické relativity. Přesná řešení však hrají důležitou roli na cestě k detailnímu porozumění obecné relativitě.

Pra naší analýzu jsou klíčové pohyby volných testovacích částic. Rovnice, které musí splňovat jejich trajektorie na prostoročasové varietě, jsou závislé na geometrii přes metrický tenzor a jeho derivace.
Pomocí nich na $\mathcal{M}$ zavádíme metrickou symetrickou (beztorzní) lineární konexi, Levi-Civitovu konexi. Pro její složky v lokálních souřadnicích platí
\begin{equation}
    \label{eq:affine_connection}
    \Gamma^\alpha_{~\mu \nu} = \frac{1}{2} g^{\alpha \sigma} \left( g_{\mu \sigma,\nu} + g_{\nu \sigma, \mu} - g_{\mu \nu, \sigma}\right),
\end{equation}
kde symbolem $g_{\alpha \beta, \gamma}$ myslíme parciální derivaci složek metriky, tedy $\partial_\gamma g_{\alpha \beta}$.

Volné (testovací) částice se pohybují po křívkách $z^\mu(\tau): \mathbb{R} \to \mathcal{M}$. Tyto křivky nazýváme geodetiky, jsou parametrizovány afinním parametrem $\tau$. Pro časupodobné geodetiky
odpovídá $\tau$ \emph{vlastnímu času}. Geodetiky splňují rovnici geodetického pohybu
\begin{equation}
    \label{eq:geodesic_equation}
    \frac{\rmd^2 z^\mu}{\rmd \tau^2} + \Gamma^\mu_{~\alpha \beta} \frac{\rmd z^\alpha}{\rmd \tau} \frac{\rmd z^\beta}{\rmd \tau}.
\end{equation}


\section*{Impulzní gravitační vlny}
V této práci se budeme zabývat prostoročasy s impulzními gravitačními vlnami. Jedná se o přesná řešení Einsteinových rovnic s křivostí úměrnou $\delta$-funkci lokalizované
na nulové nadploše (ve smyslu teorie distribucí) na které se fyzikálně nachází infinitezimálně krátký pulz silného gravitačního záření.
Mezi nejzkoumanější prostoročasy s impulzními vlnami se řadí jednoduché modely z Robinson-Trautmanovy třídy expandujících vln a z Kundtovy třídy
neexpandujících vln, respektive jejich impulzní limity, kde se profil vlny limitně blíží k $\delta$-fukci \cite{griffiths_podolsky_2009} \cite{PODOLSK2002}.
V případě expandujících i neexpandujících vln se jedná o vlny šířící se na prostorech konstantní křivosti až na nadplochu představující impulz.
Právě na ploše impulzu dochází k fyzikálně zajímavým efektům - v souřadnicích pozadí dochází ke skoku a zlomu v polohách a čtyřrychlostech
procházejících testovacích částic.

Hlavním cílem této práce je vizualizovat a blíže popsat tuto interakci testovacích částic s impulzní gravitační vlnou. Zaměříme se
zejména na neexpandující impulzní vlny Kundtovy třídy, kterým se věnuje kapitola \ref{chap:kap02}, dále v kapitole \ref{chap:kap03} na jejich zobecnění popisující impulzní vlny generované
gyratonovým zdrojem, kde má prostoročas za impulzní vlnou dodatečné mimodiagonální členy.

K vizualizacím geodetického pohybu využijeme
formalismus refrakčních rovnic, které udávají podmínky napojení geodetik před průchodem impulsní plochou a po průchodu \cite{Podolsky:2014ysa}, \cite{Podolsky_2017}.
Samotná vizualizace bude provedená za pomoci vlastního balíčku \emph{GRImpulsiveWaves} pro jazyk Python, který numericky řeší rovnici geodetiky na
částech prostoročasů před a po refrakci a následně vytváří statické i interaktivní zobrazení geodetického pohybu.

V příloze \ref{chap:kap04} nakonec přiblížíme možnost konstrukce impulsních expandujících vln pomocí spinorového popisu a představíme některé základní
dosažené výsledky studia expandujících vln spinorovým i klasickým popisem odpovídající původním pracím \cite{PenroseNutku1992}, \cite{Podolsky:2016mqg}.
Ačkoliv tento dodatek vybočuje z linie analyzované v hlavní části práce, rozhodli jsme se ho k práci
připojit jako možný zdroj inspirace k dalšímu studiu.


\clearpage