\chapter*{Úvod}
\addcontentsline{toc}{chapter}{Úvod}

\section*{Obecná teorie relativity}

Roku 1915 Albert Einstein představil a publikoval Obecnou teorii relativity \cite{Einstein1915}, nejmodernější a doposud nejúspěšnější teorii gravitace. Za více jak 100 let od její formulace byly v nesčetném množství
experimentů úspěšně ověřeny její předpovědi. První dva klíčové úspěchy této teorie byly objasnění anomálního stáčení perihelia Merkuru následované pozorováním ohybu paprsků světla procházejících v blízkosti slunce
sirem Eddingtone při zatmění slunce v roce 1919.
Technicky nejsložitější a nejdéle očekávaným testem obecné relativity byla první detekce gravitačních vln v roce 2015 na interferometrech LIGO v Livingstonu a Hanfordu, která byla následována
desítkami dalších detekcí k nimž nyní přispívá i evropský projekt VIRGO. \textcolor{red}{Další zajímavé experimentální ověření?}.


Formálně je obecná teorie relativity geometrická teorie popisující chování hmoty a energie ve čtyřrozměrném prostoročase, který reprezentuje jako diferencovatelnou varietu $\mathcal{M}$ vybavenou metrickým tenzorem $g_{\mu \nu}$.
Tvar metrického tenzoru, resp. jeho složek (metrických funkcí) je s fyzikální realitou gravitačního pole spojen Einsteinovými rovnicemi, které mají v geometrizovaných jednotkách
tvar
\begin{equation}
    \label{eq:einsten_field_equations}
    R_{\mu \nu} - \frac{1}{2} R g_{\mu \nu} + \Lambda g_{\mu \nu} = 8 \pi T_{\mu \nu},
\end{equation}
přičemž uvažujeme geometrické jednotky s $c = G = 1$.
Jedná se o 10 nelineárních parciálních diferenciálních rovnic druhého řádu, jejichž levá strana představuje geometrii na $\mathcal{M}$ reprezentující prostoročas, $R_{\mu \nu}$ je Ricciho tenzor křivosti, $R$ je Ricciho skalární křivost a $\Lambda$
je takzvaná kosmologická konstanta. Tenzor energie a hybnosti $T_{\mu \nu}$ na pravé straně pak dává spojení geometrických objektů s fyzikálním modelem, představuje rozložení hmoty,
energie a jejich toky a hybnosti. Tyto rovnice tedy říkají, jak hmota ovlivňuje geometrii prostoročasu a jak geometrie prostoročasu udává dynamiku fyzikálního obsahu prostoročasu.
Obecně je řešení této soustavy velmi obtížný problém, získání přesných řešení vyžaduje řadu zjednodušujících předpokladů, jako například
předpoklady symetrie, viz \cite{griffiths_podolsky_2009}, \cite{stephani2003}. Běžně se k řešení těchto rovnic využívá poruchových metod a metod
numerické relativity. Přesná řešení však hrají důležitou roli na cestě k detailnímu porozumění obecné relativitě.


Pomocí metrického tenzoru prostoročasové varietě zavádíme při metrickou a symetrickou (beztorzní) lineární konexi, Levi-Civitovu konexi. Pro její složky v lokálních souřadnicích platí
\begin{equation}
    \Gamma^\alpha_{~\mu \nu} = \frac{1}{2} g^{\alpha \sigma} \left( g_{\mu \sigma,\nu} + g_{\nu \sigma, \mu} - g_{\mu \nu, \sigma}\right),
\end{equation}
kde symbolem $g_{\alpha \beta, \gamma}$ myslíme parciální derivaci složek metriky $\partial_\gamma g_{\alpha \beta}$.

Volné (testovací) částice se pohybují po křívkách $z^\mu(\tau): \mathbb{R} \to \mathcal{M}$. Tyto křivky nazýváme geodetiky, jsou parametrizovány afinním parametrem $\tau$. Pro časupodobné geodetiky
odpovídá $\tau$ \emph{vlastnímu času}. Geodetiky splňují rovnici geodetického pohybu
\begin{equation}
    \frac{\rmd^2 z^\mu}{\rmd \tau^2} + \Gamma^\mu_{~\alpha \beta} \frac{\rmd z^\alpha}{\rmd \tau} \frac{\rmd z^\beta}{\rmd \tau}.
\end{equation}


\section*{Impulzní gravitační vlny}
V této práci se budeme zabývat impulzními gravitačními vlnami. Jedná se o přesná řešení Einsteinových rovnic s křivostí úměrnou $\delta$-funkci lokalizované
na nulové nadploše (ve smyslu teorie distribucí) na které se fyzikálně nachází infinitezimálně krátký pulz silného gravitačního záření.
Mezi nejzkoumanější prostoročasy s impulzními vlnami se řadí jednoduché modely z Robinson-Trautmanovy třídy expandujících vln a z Kundtovy třídy
neexpandujících vln, respektive jejich impulzní limity, kde se profil vlny limitně blíží k $\delta$-fukci.
V případě expandujících i neexpandujících vln se jedná o vlny šířící se na prostorech konstantní křivosti až na nadplochu představující impulz.
Právě na ploše impulzu dochází k fyzikálně zajímavým efektům - v souřadnicích pozadí dochází ke skoku v polohách a čtyřrychlostech
procházejících tetovacích částic.

Hlavním cílem této práce je vizualizovat a blíže popsat tuto interakci testovacích částic s impulzní gravitační vlnou. Zaměříme se
zejména na neexpandující impulzní vlny Kundtovy třídy, kterým se věnuje kapitola \ref{chap:kap02}, dále v kapitole \ref{chap:kap03} na zobecnění na impulzní vlny generované
gyratonovým zdrojem, kde má prostoročas za impulzní vlnou dodatečné mimodiagonální členy.

K vizualizacím geodetického pohybu využijeme
formalismus refrakčních rovnic, které udávají podmínky napojení geodetik před průchodem impulsní plochou a po průchodu.
Samotná vizualizace bude provedená za pomoci vlastního balíčku \emph{GRImpulsiveWaves} pro jazyk Python, který numericky řeší rovnici geodetiky na
částech prostoročasů před a po refrakci a následně vytváří statické i interaktivní zobrazení geodetického pohybu.

V příloze \ref{chap:kap04} nakonec přiblížíme možnost konstrukce impulsních expandujících vln pomocí spinorového popisu a představíme některé základní
dosažené výsledky studia expandujících vln spinorovým i klasickým popisem.


\clearpage