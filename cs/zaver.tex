\chapter*{Závěr}
\addcontentsline{toc}{chapter}{Závěr}

V této práci jsme rekapitulovali metody konstrukce prostoročasů s impulzními gravitačními vlnami
a metodu $C^1$-matchingu udávající refrakční rovnice - lepící podmínky pro geodetický pohyp při průchodu
testovací částice impulzní vlnou. Kapitola \ref{chap:kap02} se věnuje impulzní limitě neexpandujících
gravitačních vln propagujících se na Minkowského a (anti--)de Sitterově prostoročasu, v kapitole \ref{chap:kap03}
je situace zobecněna na impulsní vlny generované gyratonovým zdrojem, kde v metrice části prostoročasu "za impulsní vlnou"\
vystupují mimodiagonální členy odpovídající rotujícímu charakteru zdroje.
Refrakční rovnice byly v rámci studia této problematiky přepočítány a byl odvozen správný tvar refrakční rovnice
pro radiální rychlost v reálných polárních souřadnicích.

Hlavním původním výsledkem této práce jsou samotné vizualizace geodetického pohybu, pro které byly nejprve ve vhodných souřadnicích dopočítány
profilové funkce a jejich derivace vystupující v refrakčních rovnicích, dále byl vytvořen balíček \textit{GRImpulsiveWaves} pro programovací jazyk
Python, který za pomoci refrakčních rovnic a numerické integrace rovnice geodetiky
vytváří statické i interaktivní vizualizace geodetik v prostoročasech s impulzními vlnami.
Vizualizované chování bylo blíže popsáno a fyzikálně interpretováno.

Příloha \ref{chap:kap04} je věnována základnímu přehledu konstrukce expandujících vln pomocí spinorového formalismu.

Na tuto práci lze navázat zejména rozšířením vizualizací na případ expandujících vln. Refrakční rovnice pro
třídu řešení expandujících gravitačních vln jsou známé (například \cite{Podolsky:2016mqg}) a ukazují, že
geodetické chování při přechodu expandující vlny je složitější, než v případě vln neexpandujících. V příloze \ref{chap:kap04}
je naznačena možnost jejich odvození pomocí spinorového formalismu, což by mohlo přinést hlubší porozumění chování geodetik
při průchodu takovým impulsem.

\clearpage
