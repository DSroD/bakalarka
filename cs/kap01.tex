\chapter{Prostoročasy konstantní křivosti}
V této kapitole představáme prostoročasy, které budou sloužit jako pozadí pro propagaci impulzlních gravitačních vln. Jde o maximálně
symetrická řešení Einsteinových polních rovnic \ref{eq:einsten_field_equations} s nulovou pravou stranou a s konstantní skalární křivostí $R$ na celém prostoorčase.
Celkem rozlišujeme 3 \textcolor{red}{třídy} řešení lišící se znaménkem skalární křivosti. Nulová skalární křivost odpovídá řešení s nulovou
kosmologickou konstantou kterému se říká Minkowského prostoročas, kladná křivost odpovídá tzv. de Sitterovu prostoročasu s kladnou kosmologickou
konstantou a záporná odpovídá anti-de Sitterovu prostoročasu se zápornou kosmologickou konstantou. Vztah mezi kosmologickou konstantou a skalární
křivostí ve vakuových řešeních (tedy s nulovým tenzorem energie a hybnosti) dostaneme kontrakcí polních rovnic \ref{eq:einsten_field_equations} jako
\begin{equation}
     R = 4 \Lambda.
\end{equation}
Tato řešení jsou i přes svou jednoduchost na poli teoretické fyziky velmi podstatná \cite{Bicak:2000ea} \textcolor{red}{zde zajímavosti
o jednotlivých řešeních stručně (Minkowski - STR, AdS - CFT korespondence atp...)}
\section{Minkowskeho prostoročas}
Minkowského prostoročas je nejjednodušší řešení Einsteinových polních rovnic s nulovou pravou stranou.
Jde o řešení s nulovou křivostí v celém prostoročaseV
nejpřirozenějším vyjádření v kartézských souřadnicích má metrika prostoročasu tvar
\begin{equation}
     \label{eq:minkowski}
     \rmd s^2 = - \rmd t^2 + \rmd x^2 + \rmd y^2 + \rmd z^2.
\end{equation}
Souřadnicovými transformacemi
\begin{equation}
     \label{eq:retarded_coordinates}
     \matu = \halfsqrt (t - z),~~~~~~\matv = \halfsqrt (t + z)
\end{equation}
\begin{equation}
     \eta = \halfsqrt (x + iy), ~~~~~~\bar{\eta} = \halfsqrt (x - iy)
\end{equation}
převedeme metriku do symetrického tvaru
\begin{equation}
     \label{eq:minkowski_null_metric}
     \rmd s^2 = -2 \rmd \matu ~ \rmd \matv + 2 \rmd \eta ~ \rmd \bar{\eta}.
\end{equation}
Souřadnicím zavedeným transformací \ref{eq:retarded_coordinates} se říká retardovaná a advancovaná souřadnice,
metrika \ref{eq:minkowski_null_metric} je pak v tzv. světelných (nulových) souřadnicích.
Pokud zkoumáme axiálně symetrickou situaci na Minkowského pozadí, je vhodné zavést cylindrické
souřadnice parametrizací
\begin{equation}
     x = \rho \cos (\varphi), ~~~~~~ y = \rho \sin (\varphi),
\end{equation}
metrika pak nabývá tvaru
\begin{equation}
     \rmd s^2 = -\rmd t^2 + \rmd \rho^2 + \rho^2 \rmd \varphi^2 + \rmd z^2.
\end{equation}

Další důležitou parametrizací Minkowského prostoročasu jsou sférické souřadnice



\section{de Sitterův prostoročas}



\section{Anti-de Sitterův prostoročas}


