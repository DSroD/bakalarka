\chapter{Prostoročasy konstantní křivosti}
\label{chap:kap01}
V této kapitole představíme prostoročasy, které budou sloužit jako pozadí pro propagaci impulzlních gravitačních vln. Jde o maximálně
symetrická řešení Einsteinových polních rovnic \eqref{eq:einsten_field_equations} s nulovou pravou stranou a s konstantní skalární křivostí $R$ na celém prostoorčase.
Celkem rozlišujeme 3 \textcolor{red}{třídy} řešení lišící se znaménkem skalární křivosti. Nulová skalární křivost odpovídá řešení s nulovou
kosmologickou konstantou kterému se říká Minkowského prostoročas, kladná křivost odpovídá tzv. de Sitterovu prostoročasu s kladnou kosmologickou
konstantou a záporná odpovídá anti-de Sitterovu prostoročasu se zápornou kosmologickou konstantou. Vztah mezi kosmologickou konstantou a skalární
křivostí ve vakuových řešeních (tedy s nulovým tenzorem energie a hybnosti) dostaneme kontrakcí polních rovnic \eqref{eq:einsten_field_equations} jako
\begin{equation}
     R = 4 \Lambda.
\end{equation}
Všechna tato řešení vykazují deset symetrií, jejich interpretace jsou pro jednotlivé prostoročasy různé a budou dále diskutovány níže. \textcolor{red}{Tohle vyjádřit trochu jinak, zní to neohrabaně :D}
I přes svou jednoduchost jsou zmíněné prostoročasy na poli teoretické fyziky velmi podstatné, čtyřdimenzionální Minkowskeho prostoročas je arénou speciální
teorie relativity, vícedimenzionální Minkowského prostoročas pak běžně slouží jako vhodný prostor pro vnoření složitějších prostoročasů.
De-Sitterův prostoročas je pro moderní fyziku důležitý při popisu vesmíru, experimentální data
ukazují, že v aproximaci do prvního řádu jej lze právě na velkých škálách (a také v inflační epoše) popsat právě jako de Sitterův prostoročas.
Anti-de Sitterův prostoročas... \cite{Bicak:2000ea} \textcolor{red}{zde zajímavosti
o jednotlivých řešeních stručně (Minkowski - STR, AdS - CFT korespondence atp...)}


\section{Minkowskeho prostoročas}
Minkowského prostoročas $\mathbb{E}^{1,3}$ je nejjednodušší řešení Einsteinových polních rovnic s nulovou pravou stranou.
Jde o řešení s nulovou křivostí v celém prostoročase. Grupou symetrií Minkowského prostoročasu je Poincarého
grupa, která je tvořena všemi translacemi a Lorentzovou grupou představující rotace a boosty.

Metrika Minkowského prostoročasu má v nejpřirozenějším vyjádření, v karzézských souřadnicích, tvar
\begin{equation}
     \label{eq:minkowski}
     \rmd s^2 = - \rmd t^2 + \rmd x^2 + \rmd y^2 + \rmd z^2.
\end{equation}
Souřadnicovými transformacemi
\begin{equation}
     \label{eq:retarded_coordinates}
     \matu = \halfsqrt (t - z),~~~~~~\matv = \halfsqrt (t + z)
\end{equation}
\begin{equation}
     \eta = \halfsqrt (x + iy), ~~~~~~\bar{\eta} = \halfsqrt (x - iy)
\end{equation}
převedeme metriku do symetrického tvaru
\begin{equation}
     \label{eq:minkowski_null_metric}
     \rmd s^2 = -2 ~ \rmd \matu ~ \rmd \matv + 2 ~ \rmd \eta ~ \rmd \bar{\eta}.
\end{equation}
Souřadnicím zavedeným transformací \eqref{eq:retarded_coordinates} se říká retardovaná a advancovaná souřadnice,
metrika \eqref{eq:minkowski_null_metric} je pak v tzv. světelných (nulových) souřadnicích.
Pokud zkoumáme axiálně symetrickou situaci na Minkowského pozadí, je vhodné zavést cylindrické
souřadnice parametrizací
\begin{equation}
     x = \rho \cos (\varphi), ~~~~~~ y = \rho \sin (\varphi),
\end{equation}
kde $\rho \in [0, \infty)$ a $\varphi \in [0, 2\pi)$. Metrika pak nabývá tvaru
\begin{equation}
     \rmd s^2 = -\rmd t^2 + \rmd \rho^2 + \rho^2 \rmd \varphi^2 + \rmd z^2.
\end{equation}

V Minkowského prostoročase jsou složky afinní konexe v kartézských souřadnicích,
kterým odpovídá metrika \ref{eq:minkowski}, identicky nulové. Řešení rovnice geodetiky pak
nabývá tvaru přímek
\begin{equation}
     \label{eq:minkowski_cartesian_geodesics}
     \begin{split}
     t(\lambda) &= t_0 + \lambda {\dot t}_0 \\
     x(\lambda) &= x_0 + \lambda {\dot x}_0 \\
     y(\lambda) &= y_0 + \lambda {\dot y}_0 \\
     z(\lambda) &= z_0 + \lambda {\dot z}_0,
     \end{split}
\end{equation}
kde veličiny s dolním indexem $0$ představují počáteční podmínky.


\section{de Sitterův prostoročas}

De Sitterův prostoročas $dS_4$ je maximálně symetrické vakuové řešení Einsteinových rovnic s kladnou kosmologickou
konstantou $\Lambda$. Isometrie čtyřrozměrného de Sitterova prostoročasu tvoří grupu $SO(1,4)$. De Sitterův prostoročas topologicky odpovídá
$\mathbb{R}^1 \times \mathbb{S}^3$ a lze jej přirozeně reprezentovat jako vnoření hyperboloidu
\begin{equation}
     \label{eq:dS_hyperboloid}
     - Z_0^2 + Z_1^2 + Z_2^2 + Z_3^2 + Z_4^2 = a^2
\end{equation}
do pětidimenzionálního Minkowského prostoru $\mathbb{E}^{1,4}$ s metrikou
\begin{equation}
     \label{eq:5d_minkowski}
     \rmd s^2 = - \rmd Z_0^2 + \rmd Z_1^2 + \rmd Z_2^2 + \rmd Z_3^2 + \rmd Z_4^2.
\end{equation}
Konstanta $a$ je daná kosmologickou konstantou jako $a = \sqrt{3/\Lambda}$.

\textcolor{red}{Sem vložit vykreslený hyperboloid (TODO: v pythonu vygenerovat hyperboloid a vytvořit vektorovou grafiku, už žádné tikz)}

Hyperboloid představující de Sitterův prostoročas lze v $\mathbb{E}^{1,4}$ parametrizovat souřadnicemi $(\matu, \matv, \eta, \bar{\eta})$

\begin{equation}
     \begin{split}
          Z_0 &= \tfrac{1}{\sqrt{2}} \left(\matu + \matv\right) \left[1-\tfrac{1}{6} \Lambda \left(\matu \matv - \eta \bar{\eta}\right)\right]^{-1} \\
          Z_1 &= \tfrac{1}{\sqrt{2}} \left(\matu - \matu\right) \left[1-\tfrac{1}{6} \Lambda \left(\matu \matv - \eta \bar{\eta}\right)\right]^{-1} \\
          Z_2 &= \tfrac{1}{\sqrt{2}} \left(\eta + \bar{\eta}\right) \left[1-\tfrac{1}{6} \Lambda \left(\matu \matv - \eta \bar{\eta}\right)\right]^{-1} \\
          Z_3 &= \tfrac{1}{\sqrt{2}} \left(\eta - \bar{\eta}\right) \left[1-\tfrac{1}{6} \Lambda \left(\matu \matv - \eta \bar{\eta}\right)\right]^{-1} \\
          Z_4 &= a \left[1 + \tfrac{1}{6} \Lambda \left(\matu \matv - \eta \bar{\eta}\right)\right] \left[1-\tfrac{1}{6} \Lambda \left(\matu \matv - \eta \bar{\eta}\right)\right]^{-1},
     \end{split}
\end{equation}
které pro $\matu, \matv \in \left(-\infty, \infty\right)$ a komplexní $\eta$ pokrývají de Sitterův hyperboloid globálně, až na singularity v  $\mathcal{U}, \mathcal{V}=\textcolor{red}{?\pm}\infty$.
Tyto souřadnice indukují metriku
\begin{equation}
     \label{eq:konfmetric}
     \rmd s^2 = \frac{-2 ~ \rmd \matu ~ \rmd \matv + 2 ~ \rmd \eta ~ \rmd \bar{\eta}}{\left[1-\tfrac{1}{6} \Lambda \left(\matu \matv - \eta \bar{\eta}\right)\right]^2}.
\end{equation}
Tato metrika je v takzvaném konformně plochém tvaru, tedy ve tvaru
\begin{equation}
     \rmd s^2 = \frac{\rmd s_0^2}{\Omega^2}
\end{equation}
kde $\rmd s_0^2$ je metrika na $\mathbb{E}^{1,3}$ a $\Omega$ je hladkou funkcí souřadnic \textcolor{red}{Jaké jsou na $\Omega$ kladeny podmínky (aby se opravdu jednalo o konformní souřadnice)}

\textcolor{red}{Další pokrytí hyperboloidu (třeba FLRW metrika?)}


\section{Anti-de Sitterův prostoročas}

Anti-de Sitterův prostoročas $AdS_4$ je maximálně symetrické vakuové řešení Einsteinových rovnic se zápornou
kosmologickou konstantou $\Lambda$. Isometrie tvoří grupu $SO(2,3)$ a topologie $AdS_4$ odpovídá $\mathbb{S^1} \times \mathbb{R}^3$.
Vnořením $AdS_4$ do $\mathbb{E}^{2,3}$, tedy do prostoru s metrikou
\begin{equation}
     \rmd s^2 = -\rmd Z_0^2 + \rmd Z_1^2 + \rmd Z_2^2 + \rmd Z_3^2 - \rmd Z_4^2,
\end{equation}
vzniká hyperboloid
\begin{equation}
     -Z_0^2 + Z_1^2 + Z_2^2 + Z_3^2 - Z_4^2 = a^2,
\end{equation}
kde $a = -\sqrt{\Lambda/3}$.

\textcolor{red}{Stejně jako u dS}