\chapter{Prostoročasy konstantní křivosti}
V této kapitole představáme prostoročasy, které budou sloužit jako pozadí pro propagaci impulzlních gravitačních vln. Jde o maximálně
symetrická řešení Einsteinových polních rovnic \ref{eq:einsten_field_equations} s nulovou pravou stranou a s konstantní skalární křivostí $R$ na celém prostoorčase.
Celkem rozlišujeme 3 \textcolor{red}{třídy} řešení lišící se znaménkem skalární křivosti. Nulová skalární křivost odpovídá řešení s nulovou
kosmologickou konstantou kterému se říká Minkowského prostoročas, kladná křivost odpovídá tzv. de Sitterovu prostoročasu s kladnou kosmologickou
konstantou a záporná odpovídá anti-de Sitterovu prostoročasu se zápornou kosmologickou konstantou. Vztah mezi kosmologickou konstantou a skalární
křivostí ve vakuových řešeních (tedy s nulovým tenzorem energie a hybnosti) dostaneme kontrakcí polních rovnic \ref{eq:einsten_field_equations} jako
\begin{equation}
     R = 4 \Lambda.
\end{equation}
Tato řešení jsou i přes svou jednoduchost na poli teoretické fyziky velmi podstatná \cite{Bicak:2000ea} \textcolor{red}{zde zajímavosti
o jednotlivých řešeních stručně (Minkowski - STR, AdS - CFT korespondence atp...)}
\section{Minkowskeho prostoročas}



\section{De Sitterův prostoročas}



\section{Anti-de Sitterův prostoročas}


