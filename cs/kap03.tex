\chapter{Neexpandující gravitační vlny s gyratonovými členy}
\label{chap:kap03}
Distribuční vyjádření metriky impulzní vlny \eqref{eq:nonexp_distr_metric_omega} není nejobecnější metrikou popisující
řešení toho typu. Lze jí rozšířit o mimodiagonální členy do tvaru
\begin{equation}
    \label{eq:nonexp_gyra_distrib_metric_omega}
    \begin{split}
        \mathrm{d}s^2=&\frac{2\mathrm{d}\eta~\mathrm{d}\bar{\eta} - 2 \mathrm{d}\mathcal{U}~\mathrm{d}\mathcal{V} + 2H(\eta, \bar{\eta}) \delta(\mathcal{U}) 
        ~\mathrm{d}\mathcal{U}^2}{\left[1+\frac{1}{6}\Lambda(\eta \bar{\eta}-\mathcal{U}\mathcal{V})\right]^2} \\
        &+ \frac{2J\left(\eta, \bar{\eta}, \mathcal{U}\right) \mathrm{d}\eta~\mathrm{d}\mathcal{U}
        +2\overline{J}\left(\eta, \bar{\eta}, \mathcal{U}\right) \mathrm{d}\bar{\eta}~\mathrm{d}\mathcal{U}}{\left[1+\frac{1}{6}\Lambda(\eta \bar{\eta}-\mathcal{U}\mathcal{V})\right]^2}.
    \end{split}
\end{equation}
Tvar metriky s mimodiagonálními členy odpovídající gravitační vlně byl uvažován už v původní Brinkmannově studii \cite{Brinkmann1925} Einsteinových prostoročasů svázaných konformními transformacemi.

Obvyklým postupem je odstranění členů s funkcí $J$ vhodnou souřadnicovou transformací (Brinkmannova forma metriky \emph{pp}-vln \cite{griffiths_podolsky_2009} se také běžně uvádí už bez mimodiagonálních členů), to ale vede k odstranění
možného rotačního charakteru zdroje gravitační vlny. Taková transformace pak není globální a dochází ke změně topologických vlastností celého prostoročasu.
Rotující charakter zdrojů byl zkoumán Bonnorem \cite{Bonnor1970} a nezávisle objeven Frolovem a Fursaevem v \cite{Frolov2005_0} a \cite{Frolov2005}. Přehled a podrobnější
fyzikální analýzu je možné najít v \cite{Podolsky2014}.
V této kapitole budeme studovat a vizualizovat právě vliv nediagonálních prvků metriky.

\section{Zobecnění impulzních vln na prostoročasy s~gyratonovými členy}
Konstrukce neexpandujících gracitačních vln s mimodagonálními gyratonovými členy je obdobná konstrukci prostoročasů bez gyratonů z minulé kapitoly. Popíšeme ji formalismem spojitých souřadnic
použitým v \cite{Podolsky_2017}, kde jsou také odvozeny refrakční rovnice pro pohyb testovacích částic, pomocí kterých budeme vizualizovat geodetický pohyb procházející impulzní vlnoplochou.

Penroseova "cut and paste"\ konstrukce vede i v případě gyratonových prostoročasů na Penroseovy lepící podmínky ve tvaru \eqref{eq:lepici_podminky}, začneme tedy rovnou zobecněním
spojitého tvaru metriky z předchozí kapitoly.

\subsection{Zobecnění spojitého tvaru metriky}
Tvar spojité metriky pro gyratonové neexpandující impulzní vlny obdržíme zobecněním transformace \eqref{eq:nonexp_cont_full_transform},
do tvaru který nalezli Podolský, Švarc, Sämann a Steinbauer v \cite{Podolsky_2017},
\begin{equation}
    \label{eq:gyraton_cont_transformation}
    \begin{split}
        \matu &= U, \\
        \matv &= V + \Theta H + U_{+} H_{,Z} H_{,\bar{Z}} + W, \\
        \eta &= \left(Z + U_{+} H_{,\bar{Z}}\right) \exp \left(i F \right),
    \end{split}
\end{equation}
kde opět $H = H(Z, \bar{Z})$, zároveň funkce $W = W(Z, \bar{Z}, U)$ a $F = F(Z, \bar{Z}, U)$ jsou reálné a splňují
\begin{equation}
    \label{eq:podminky_F_a_W}
    \begin{split}
        F_{,U} &= \frac{i\bar{J}}{Z + U_{+}H_{,\bar{Z}}} \exp{\left(-iF\right)}, \\
        W_{,U} &= -J \bar{J}, \\
        W &= 0 \text{ pro } U \leq 0.
    \end{split}
\end{equation}
Zavedením
\begin{equation}
    \zeta \equiv Z + U_{+} H_{,\bar{Z}}
\end{equation}
a prostorového diferenciálu $\underline{\rmd}$ tak, že
\begin{equation}
    \label{eq:spatial_differential_on_functions}
    \begin{split}
        \underline{\rmd}\zeta &\equiv \rmd Z + U_{+}(H_{,\bar{Z}Z}\rmd Z + H_{,\bar{Z}\bar{Z}}\rmd \bar{Z}), \\
        \underline{\rmd}H &\equiv H_{,Z}\rmd Z + H_{,\bar{Z}}\rmd \bar{Z}, \\
        \underline{\rmd}F &\equiv F_{,Z}\rmd Z + F_{,\bar{Z}}\rmd \bar{Z}, \\
        \underline{\rmd}W &\equiv W_{,Z}\rmd Z + W_{,\bar{Z}}\rmd \bar{Z},
    \end{split}
\end{equation}
spolu s dosazením \eqref{eq:gyraton_cont_transformation} do \eqref{eq:nonexp_gyra_distrib_metric_omega} a využitím vztahů \eqref{eq:podminky_F_a_W},
\eqref{eq:spatial_differential_on_functions} a multiplikativních pravidel z nelineární teorie distribucí,
\begin{equation}
    \label{eq:pravidla_distribuce}
    \Theta^2 = \Theta, ~~~~~~ \Theta U_{+} = U_{+},
\end{equation}
dostaneme metriku ve spojitém tvaru
\begin{equation}
    \label{eq:spojita_gyratonova_metrika}
    \rmd s^2 = \frac{2 \left|\underline{\rmd} \zeta + i \zeta \underline{\rmd}F\right|^2 + 2 \left[i \Theta\left(\zeta H_{,Z} - \bar{\zeta} H_{,\bar{Z}}\right) \underline{\rmd}F - \underline{\rmd}W\right] \rmd U - 2\rmd U \rmd V}{\left[1+\frac{1}{6}\Lambda\left(Z \bar{Z} - UV - U_{+}G\right)\right]^2},
\end{equation}
kde byla zavedena funkce
\begin{equation}
    G \left( Z, \bar{Z}, U \right) = H - Z H_{,Z} - \bar{Z} H_{,\bar{Z}} + W.
\end{equation}

Oproti minulé kapitole se ve funkci $G$ vyskytuje člen $W$. Aplikací transformace \eqref{eq:gyraton_cont_transformation}
na konformní faktor $\Omega = 1 + \frac{1}{6} \Lambda (\eta \bar{\eta} - \matu \matv)$ dostáváme po použití pravidel násobení distribucí v nelineární teorii \eqref{eq:pravidla_distribuce} tvar
\begin{equation}
    \Omega = 1 + \frac{1}{6} \Lambda \left(Z \bar{Z} - UV - U_+ (H - Z H_{,Z} \bar{Z} H_{,\bar{Z}}) - UW \right).
\end{equation}
V případě že $W = 0$ pro $U \leq 0$, platí $UW = U_+ W$. To je ale splněno v \eqref{eq:podminky_F_a_W} a tvar funkce $G$ je tedy ospravedlněn.

Volba $J=0$ dovoluje řešení $F = W = 0$, přičemž se metrika \eqref{eq:spojita_gyratonova_metrika} redukuje na \eqref{eq:nonexp_continuous_metric}.
Metrika \eqref{eq:spojita_gyratonova_metrika} je spojitá v případě, že $\underline{\rmd}F$ a $\underline{\rmd}W$ jsou funkce spojité v $U$ a $\underline{\rmd}F$
jde v $U=0$ k nule. V tomto případě je metrika lokálně lipschitzovská a lze využít formalismu Filippovových řešení jako v předchozí kapitole.


\subsection{Frolovův--Fursaevův gyraton}
Dále budeme uvažovat konkrétní realizaci gyratonového prostoročasu, tedy metriku \eqref{eq:nonexp_gyra_distrib_metric_omega} s volbou
\begin{equation}
    J(\eta , \matu) = \frac{\chi}{2 i \eta} \Theta( \matu ),~~~ \Lambda = 0,
\end{equation}
kde $\chi$ je konstantní. Rovnice \eqref{eq:podminky_F_a_W} lze zintegrovat do tvaru
\begin{equation}
    \label{eq:zintegrovane1}
    \begin{split}
        F &= \frac{\chi}{2(Z H_{,Z}-\bar{Z}H_{,\bar{Z}})} \log \frac{Z\bar{Z}+U_{+}\bar{Z}H_{,\bar{Z}}}{Z\bar{Z}+U_{+}ZH_{,Z}}, \\
        W &= \frac{\chi}{2}F,
    \end{split}
\end{equation}
kde přirozeně uvažujeme hlavní větev logaritmu, aby se zachovala rovnost $\eta = Z$ pro $U \leq 0$.

V případě, že $Z H_{,Z} - \bar{Z}H_{,\bar{Z}}=0 \iff \log \frac{Z \bar{Z} + U_{+}\bar{Z}H_{,\bar{Z}}}{Z\bar{Z}+U_{+}ZH_{,Z}}=0$
není tvar \eqref{eq:zintegrovane1} platný. Diferenciální operátor působící na funkci $H$ v levé části ekvivalence
lze (díky $Z = \frac{1}{\sqrt{2}} \rho \exp(i \phi)$) zapsat jako $Z \partial_Z - \bar{Z}\partial_{\bar{Z}} = - i \partial_\phi$,
jedná se tedy o generátor rotace kolem osy $Z = 0$. Tento případ tedy nastává ve všech axiálně symetrických prostoročasech, kde
funkce $H$ závisí pouze na $\rho^2 = 2 Z \bar{Z}$, včetně Aichelburgova--Sexlova řešení
$H = b_0 \log(2\eta \bar{\eta}) = b_0 \log (2Z \bar{Z})$, kterým se budeme dále zabývat.
Funkce $F$ a $W$ lze v takovém případě volit ve tvaru
\begin{equation}
    \label{eq:zintegrovane2}
    \begin{split}
        F &= -\frac{\chi}{2} \frac{U_{+}}{Z \bar{Z} + b_0 U_{+}}, \\
        W &= \frac{\chi}{2}F.
    \end{split}
\end{equation}
Spojitá metrika pak nabývá tvaru z třídy Frolovových--Fursaevových gyratonů \cite{Frolov2005} 
\begin{equation}
    \begin{split}
    \rmd s^2 = &2 \left| \rmd Z + U_{+} \left(i \frac{\chi}{2}\frac{\bar{Z}\rmd Z + Z \rmd \bar{Z}}{\bar{Z}\left(Z \bar{Z} + b_0 U_{+}\right)} - b_0 \frac{d\bar{Z}}{\bar{Z}^2}\right) \right|^2 \\
    &- \frac{\chi^2}{2}U_{+}\frac{\bar{Z} \rmd Z \rmd U + Z \rmd \bar{Z} \rmd U}{\left(Z \bar{Z} + b_0 U_{+}\right)^2} - 2 \rmd U \rmd V
    \end{split}
\end{equation}
reprezentujících zobecnění
originální Aichelburgovy--Sexlovy metriky \cite{Aichelburg_1971} na impulzní vlnu generovanou částicí s
nenulovým vnitřním momentem hybnosti, přičemž $\chi \Theta(\matu)$ odpovídá hustotě momentu hybnosti gyratonu \cite{Podolsky2014}.
Při absenci gyratonu ($\chi=0$) se metrika redukuje na standardní Aichelburgovo--Sexlovo řešení.

Díky tomu, že pracujeme v konformně plochých souřadnicích, nejsou výše uvedené závěry závislé na kosmologické konstantě.
Za funkci $H$ pak můžeme uvažovat například Hottovo--Tanakovo řešení \eqref{eq:Hotta_Tanaka_H_conf_flat_coords}. Pro toto řešení nebudeme explicitně odvozovat
spojitý tvar metriky, později ale využijeme formalismus refrakčních rovnic pro vykreslení geodetického pohybu gyratonového zdroje na (anti-)de Sitterově prostoročasu
s výše uvedenou funkcí $J$ a funkcí $H$ odpovídající právě Hottově--Tanakově řešení. Zmíníme ale, že ekvivalence znemožňující integraci funkcí $F$ a $W$ do tvaru
\eqref{eq:zintegrovane1} neplatí ani v tomto případě, jedná se také o axiálně symetrické řešení.


\section{Refrakční rovnice pro geodetiky v impulzních gyratonových prostoročasech}
Stejně jako v případě bez gyratonových členů v předchozí kapitole, i zde k vizualizaci geodetik využijeme refrakčních rovnic,
které byly pro impulzní neexpandující gyratonové prostoročasy s funkcemi $F$ a $W$ ve tvaru
\eqref{eq:zintegrovane1}, resp. \eqref{eq:zintegrovane2}, odvozeny z limity spojitých souřadnic \eqref{eq:spojita_gyratonova_metrika}
v $\matu = 0$ v článku \cite{Podolsky_2017}.
Napojovací podmínky prostorových poloh jsou v tomto případě ve tvaru
\begin{equation}
    \label{eq:refrakcni_rovnice_gyra_polohy}
    \begin{split}
        \matu_{\rmi}^{+} &= \matu_{\rmi}^{-}, \\
        \matv_{\rmi}^{+} &= \matv_{\rmi}^{-} + H_{\rmi}, \\
        \eta_{\rmi}^{+} &= \eta_{\rmi}^{-},
    \end{split}
\end{equation}
což přesně odpovídá Penroseovým napojovacím podmínkám \eqref{eq:lepici_podminky},
jejichž tvar nezávisí na přítomnosti gyratonových členů. Co se bude lišit od
výsledků popsaných v minulé kapitole jsou refrakční rovnice pro složky rychlosti,
\begin{equation}
    \label{eq:refrakcni_rovnice_gyra_rychlosti}
    \begin{split}
        \dot{\matu}_{\rmi}^{+} &= \dot{\matu}_{\rmi}^{-}, \\
        \dot{\matv}_{\rmi}^{+} &= \dot{\matv}_{\rmi}^{-} + H_{\rmi, Z} \dot{\eta}_{\rmi}^{-} + H_{\rmi, \bar{Z}} \dot{\bar{\eta}}_{\rmi}^{-} + \left(H_{\rmi, Z} H_{\rmi, \bar{Z}} - \frac{\chi^2}{4 \eta_{\rmi}^{-} \bar{\eta}_{\rmi}^{-}}\right) \dot{\matu}_{\rmi}^{-}, \\
        \dot{\eta}_{\rmi}^{+} &= \dot{\eta}_{\rmi}^{-} + \left(H_{\rmi, \bar{Z}} - \frac{i \chi}{2 \bar{\eta}_{\rmi}^{-}}\right) \dot{\matu}_{\rmi}^{-},
    \end{split} 
\end{equation}
kde gyratonové členy s $\chi$ přispívají ke skoku ve složkách $\dot{\matv}_{\rmi}$ a $\dot{\eta}_{\rmi}$ členy
\begin{equation}
    \Delta \dot{\matv}_{\rmi} = -\frac{\chi^2}{4 \eta_{\rmi}^{-} \bar{\eta}_{\rmi}^{-}} \dot{\matu}_{\rmi}^{-}, ~~~~~~ \Delta \dot{\eta}_{\rmi} = -\frac{i \chi}{2 \bar{\eta}_{\rmi}^{-}} \dot{\matu}_{\rmi}^{-}.
\end{equation}
Tyto skoky navíc se díky $C^1-$regularitě geodetik a spojitosti metriky kompenzují, neboť normalizace čtyřrychlosti musí být zachována, a platí
\begin{equation}
    \Delta \dot{\eta}_{\rmi} \Delta \dot{\bar{\eta}}_{\rmi} = - \dot{\matu}_{\rmi} \Delta \dot{\matv}_{\rmi}.
\end{equation}
Dále si povšimněme, že tyto skoky pro $\eta_{\rmi}^{-} \to 0$ rostou nad všechny meze. Toto chování ale očekáváme -- v $\eta_{\rmi} = 0$ se nachází bodový zdroj.
Pro $\chi \to 0$ se tyto rovnice redukují na rovnice \eqref{eq:refraction_nonexpanding_velocities} z minulé kapitoly.

Obdobně jako v předchozí kapitole, refrakční rovnice \eqref{eq:refrakcni_rovnice_gyra_rychlosti} můžeme také zapsat v reálných
polárních souřdnicích jako
\begin{equation}
    \begin{split}
    \dot{\matv}_{\rmi}^{+} &= \dot{\matv}_{\rmi}^{-} + \halfsqrt \left(e^{i \varphi_{\rmi}^{-}} H_{\rmi, Z} + e^{-i \varphi_{\rmi}^{-}} H_{\rmi, \bar{Z}}\right) \dot{\rho}_{\rmi}^{-} +
    \frac{i}{\sqrt{2}} \left( e^{i \varphi_{\rmi}^{-}} H_{\rmi, Z} - e^{-i \varphi_{\rmi}^{-}} H_{\rmi, \bar{Z}} \right) \rho_{\rmi}^{-} \dot{\varphi}_{\rmi}^{-} \\
    &~~~+ \left(H_{\rmi, Z} H_{\rmi, \bar{Z}} - \frac{\chi^2}{2(\rho_{\rmi}^{-})^2}\right)\dot{\matu}_{\rmi}^{-}, \\
    \dot{\rho}_{\rmi}^{+} &= \dot{\rho}_{\rmi}^{-} + \halfsqrt \left(e^{-i \varphi_{\rmi}^{-}} H_{\rmi, \bar{Z}} + e^{i \varphi_{\rmi}^{-}} H_{\rmi, Z} \right), \\
    \dot{\varphi}_{\rmi}^{+} &= \dot{\varphi}_{\rmi}^{-} + \left[ \frac{i}{\sqrt{2} \rho_{\rmi}^{-}} \left(e^{i \varphi_{\rmi}^{-}} H_{\rmi, Z} - e^{-i \varphi_{\rmi}^{-}} H_{\rmi, \bar{Z}} \right) - \frac{\chi}{(\rho_{\rmi}^{-})^2}\right] \dot{\matu}_{\rmi}^{-}.
    \end{split}
\end{equation}
Tento tvar potvrzuje interpretaci gyratonových členů jako interního momentu hybnosti částice generující impulz -- 
přítomnost gyraonických členů nemá vliv na radiální složku $\dot{\rho}$, přispívá ale k další změně v
axiální složce $\dot{\varphi_{\rmi}}$ členem $\frac{\chi}{(\rho_{\rmi})^2}$.

\section{Vizualizace geodetik v neexpandujících impulzních gyratonových prostoročasech}
V této části budeme explicitně analyzovat geodetický pohyb v prostoročasech s gyratonovými zdroji, generujícími impulzní vlny.
\subsection{Geodetický pohyb ve Frolovově--Fursaevově gyratonovém řešení}

Stejně jako v minulé kapitole využijeme refrakční rovnice \eqref{eq:refrakcni_rovnice_gyra_polohy}, \eqref{eq:refrakcni_rovnice_gyra_rychlosti}
k vizualizaci interakce impulzního Frolovova--Fursaevova gyratonu s geodetikami na prostoročasech tvořících pozadí impulzní vlny.
V případě gyratonu je pozadí odlišné, pro $\Lambda = 0$ je prostoročas $\mathcal{M}^{-}$ (před impulzní vlnou) Minkowského prostoročas,
za impulzní vlnou je ale pozadí $\mathcal{M}^+$ popsáno metrikou
\begin{equation}
    \rmd s^2 = - 2 \rmd \matu ~ \rmd \matv + 2 \rmd \eta ~ \rmd \bar{\eta} + i \chi \rmd \matu \left(\frac{\rmd \bar{\eta}}{\bar{\eta}} - \frac{\rmd \eta}{\eta}\right),
\end{equation}
jediné nenulové Christoffelovy symboly jsou
\begin{equation}
    \Gamma^{\matv}_{\eta \eta} = - \frac{i \chi}{2 \eta^2}, ~~~~~ \Gamma^{\matv}_{\bar{\eta}\bar{\eta}} = \frac{i \chi}{2 \bar{\eta}^2}.
\end{equation}

Rotující charakter zdroje je také dobře vidět z metriky za impulzní vlnou při parametrizaci reálnými souřadnicemi $(x, y)$, viz. \eqref{eq:complex_coordinates},
ve kterých metrika nabývá tvaru
\begin{equation}
    \label{eq:gyra_plus_uv_metric}
    \rmd s^2 = -2 \rmd \matu ~ \rmd \matv + \rmd x^2 + \rmd y^2 + \frac{2 \chi \rmd \matu (x~\rmd y - y~\rmd x)}{x^2 + y^2}.
\end{equation}
Transformací do standardních polárních souřadnic $x = r \sin{\varphi}, y = r \cos{\varphi}$ přechází metrika \eqref{eq:gyra_plus_uv_metric} do tvaru
\begin{equation}
    \rmd s^2 = \rmd r^2 + r^2 \rmd \varphi^2 - 2 \rmd \matu (\rmd \matv + \chi \rmd \varphi).
\end{equation}
Prostoročas za vlnou se tak netriviálně liší od maximálně symetrického pozadí (nejedná se ani o statické řešení) a zvolený přístup
numerické integrace rovnice geodetiky již v tomto nejjednodušším případě začíná dostávat plný význam.

Na následujících vizualizacích je pro různé hodnoty parametru $\chi$ znázorněno chování nejprve nulových a následně časupodobných geodetik procházejících impulzní plochou v
axiálně symetrickém uspořádání kolem gyratonového zdroje na nadploše $\matu = 0, \matv=0$ v různé vzdálenosti $\rho$ od osy symetrie.

Na obrázku \ref{fig:gyra_flat_null_b1_chi1_2} jsou pro hodnoty $\chi = \pi, 2\pi$ vizualizovány nulové geodetiky procházející nadplochou $\matu = 0$ s
čtyřrychlostí $\dot{\matu} = 1, \dot{\matv} = 0, \dot{\eta} = 0$. Oproti negyratonovému řešení, zkoumanému v předchozí kapitole, dochází
nejen ke strhnutí geodetik k ose symetrie, ale i k vychýlení vlivem rotačního charakteru zdroje. V této vizualizaci se
přítomnost gyratonu projevuje pouze refrakcí na impulzní nadploše -- jediné nenulové Christoffelovy symboly vstupují do geodetické rovnice pro $\matv$.
S~rostoucím parametrem $\chi$ je strhávání silnější.
\begin{figure}[H]
    \centering
    \begin{subfigure}[b]{0.48\textwidth}
        \begin{tikzpicture}
            \node[inner sep=0pt, anchor=south west] (ds) at (0,0)
            {\adjincludegraphics[trim={{.17\width} {.2\height} {.17\width} {.22\height}}, width=1\textwidth, clip]{../img/kap03/flat_gyraton/null/null_gyraton_ring__r_2__mu_1__chi_1pi_uxy.pdf}};
            \filldraw[white] (0.45,2.8) circle (5pt);
            \node[text width=7pt] at (0.45,2.8) {\footnotesize{$\matu$}};
            \filldraw[white] (2,0.55) circle (6pt);
            \node[text width=7pt] at (2,0.55) {\footnotesize{$y$}};
            \filldraw[white] (5.68,0.88) circle (6pt);
            \node[text width=7pt] at (5.68,0.88) {\footnotesize{$x$}};
            \filldraw[white] (0.82,1.13) circle (4pt);
        \end{tikzpicture}
         \caption{$b_0=1, \chi=\pi, \rho=2\sqrt{2}$} 
    \end{subfigure}
    \begin{subfigure}[b]{0.48\textwidth}
        \begin{tikzpicture}
            \node[inner sep=0pt, anchor=south west] (ds) at (0,0)
            {\adjincludegraphics[trim={{.17\width} {.2\height} {.17\width} {.22\height}}, width=1\textwidth, clip]{../img/kap03/flat_gyraton/null/null_gyraton_ring__r_3__mu_1__chi_1pi_uxy.pdf}};
            \filldraw[white] (0.45,2.8) circle (5pt);
            \node[text width=7pt] at (0.45,2.8) {\footnotesize{$\matu$}};
            \filldraw[white] (2,0.55) circle (6pt);
            \node[text width=7pt] at (2,0.55) {\footnotesize{$y$}};
            \filldraw[white] (5.68,0.88) circle (6pt);
            \node[text width=7pt] at (5.68,0.88) {\footnotesize{$x$}};
        \end{tikzpicture}
         \caption{$b_0=1, \chi=\pi, \rho=3\sqrt{2}$} 
    \end{subfigure}
    \hfill
    \begin{subfigure}[b]{0.48\textwidth}
        \begin{tikzpicture}
            \node[inner sep=0pt, anchor=south west] (ds) at (0,0)
            {\adjincludegraphics[trim={{.17\width} {.2\height} {.17\width} {.22\height}},width=1\textwidth, clip]{../img/kap03/flat_gyraton/null/null_gyraton_ring__r_2__mu_1__chi_2pi_uxy.pdf}};
            \filldraw[white] (0.45,2.8) circle (5pt);
            \node[text width=7pt] at (0.45,2.8) {\footnotesize{$\matu$}};
            \filldraw[white] (2,0.55) circle (6pt);
            \node[text width=7pt] at (2,0.55) {\footnotesize{$y$}};
            \filldraw[white] (5.68,0.88) circle (6pt);
            \node[text width=7pt] at (5.68,0.88) {\footnotesize{$x$}};
        \end{tikzpicture}
         \caption{$b_0=1, \chi=2\pi, \rho=2\sqrt{2}$} 
    \end{subfigure}
    \begin{subfigure}[b]{0.48\textwidth}
        \begin{tikzpicture}
            \node[inner sep=0pt, anchor=south west] (ds) at (0,0)
            {\adjincludegraphics[trim={{.17\width} {.2\height} {.17\width} {.22\height}}, width=1\textwidth, clip]{../img/kap03/flat_gyraton/null/null_gyraton_ring__r_3__mu_1__chi_2pi_uxy.pdf}};
            \filldraw[white] (0.45,2.8) circle (5pt);
            \node[text width=7pt] at (0.45,2.8) {\footnotesize{$\matu$}};
            \filldraw[white] (2,0.55) circle (6pt);
            \node[text width=7pt] at (2,0.55) {\footnotesize{$y$}};
            \filldraw[white] (5.68,0.88) circle (6pt);
            \node[text width=7pt] at (5.68,0.88) {\footnotesize{$x$}};
        \end{tikzpicture}
         \caption{$b_0=1, \chi=2\pi, \rho=3\sqrt{2}$} 
    \end{subfigure}
    \caption{Nulové geodetiky procházející impulzní plochou Frolovova--Fursaevova gyratonu, vykreslené souřadnice $\matu, x, y$.}
    \label{fig:gyra_flat_null_b1_chi1_2}
\end{figure}

\begin{figure}[H]
    \centering
    \begin{subfigure}[b]{0.48\textwidth}
        \begin{tikzpicture}
            \node[inner sep=0pt, anchor=south west] (ds) at (0,0)
            {\adjincludegraphics[trim={{.17\width} {.2\height} {.17\width} {.22\height}}, width=1\textwidth, clip]{../img/kap03/flat_gyraton/null/null_gyraton_ring__r_1__mu_1__chi_2pi_vxy.pdf}};
            \filldraw[white] (0.45,2.8) circle (5pt);
            \node[text width=7pt] at (0.45,2.8) {\footnotesize{$\matv$}};
            \filldraw[white] (2,0.55) circle (6pt);
            \node[text width=7pt] at (2,0.55) {\footnotesize{$y$}};
            \filldraw[white] (5.68,0.88) circle (6pt);
            \node[text width=7pt] at (5.68,0.88) {\footnotesize{$x$}};
        \end{tikzpicture}
        \caption{$b_0=1, \chi=2\pi, \rho=\sqrt{2}$}
    \end{subfigure}
    \begin{subfigure}[b]{0.48\textwidth}
        \begin{tikzpicture}
            \node[inner sep=0pt, anchor=south west] (ds) at (0,0)
            {\adjincludegraphics[trim={{.17\width} {.2\height} {.17\width} {.22\height}}, width=1\textwidth, clip]{../img/kap03/flat_gyraton/null/null_gyraton_ring__r_2__mu_1__chi_2pi_vxy.pdf}};
            \filldraw[white] (0.45,2.8) circle (5pt);
            \node[text width=7pt] at (0.45,2.8) {\footnotesize{$\matv$}};
            \filldraw[white] (2,0.55) circle (6pt);
            \node[text width=7pt] at (2,0.55) {\footnotesize{$y$}};
            \filldraw[white] (5.68,0.88) circle (6pt);
            \node[text width=7pt] at (5.68,0.88) {\footnotesize{$x$}};
        \end{tikzpicture}
        \caption{$b_0=1, \chi=2\pi, \rho=2\sqrt{2}$} 
    \end{subfigure}
    \caption{Nulové geodetiky po průchodu impulzní plochou Frolovova--Fursaevova gyratonu, vykreslené souřadnice $\matv, x, y$.}
    \label{fig:gyra_flat_null_b1_chi1_2_vxy}
\end{figure}

Vliv gyratonového charakteru prostoročasu za impulzní plochou je vidět při vizualizaci souřadnice $\matv$ (viz obrázek \ref{fig:gyra_flat_null_b1_chi1_2_vxy}),
při refrakci ve vykresleném případě kladná složka $\dot{\matv}$ přechází v zápornou, vlivem gyratonu je ale vynucováno kladné zrychlení a
geodetika se ve směru $\matv$ obrací.

Na obrázku \ref{fig:gyra_skewed_flat_null_b1_chi2} pak vidíme nulové geodetiky procházející impulzní nadplochou
s čtyřrychlostí $\dot{\matu} = 1, \dot{\matv} = 1, \dot{\eta} = 1$, před průchodem $\matu=0$ se v Minkowského části
prostoročasu jednotlivé geodetiky pohybují na nadploše $t, x$.

\begin{figure}[H]
    \centering
    \begin{subfigure}[b]{1\textwidth}
        \begin{subfigure}[b]{0.48\textwidth}
            \begin{tikzpicture}
                \node[inner sep=0pt, anchor=south west] (ds) at (0,0)
                {\adjincludegraphics[trim={{.14\width} {.15\height} {.15\width} {.22\height}}, width=1\textwidth, clip]{../img/kap03/flat_gyraton/null/null_skewed_gyraton_ring__r_1__mu_1__chi_2pi_uxy.pdf}};
                \filldraw[white] (0.39,2.91) circle (5pt);
                \node[text width=7pt] at (0.39,2.91) {\footnotesize{$\matu$}};
                \filldraw[white] (2.2,0.4) circle (6pt);
                \node[text width=7pt] at (2.2,0.4) {\footnotesize{$x$}};
                \filldraw[white] (5.87,1.14) circle (6pt);
                \node[text width=7pt] at (5.87,1.14) {\footnotesize{$y$}};
            \end{tikzpicture}
        \end{subfigure}
        \hfill
        \begin{subfigure}[b]{0.48\textwidth}
            \begin{tikzpicture}
                \node[inner sep=0pt, anchor=south west] (ds) at (0,0)
                {\adjincludegraphics[trim={{.12\width} {.1\height} {.15\width} {.18\height}}, width=1\textwidth, clip]{../img/kap03/flat_gyraton/null/null_skewed_top_gyraton_ring__r_1__mu_1__chi_2pi_uxy.pdf}};
                \filldraw[white] (0.39,3.4) circle (5pt);
                \node[text width=7pt] at (0.39,3.4) {\footnotesize{$\matu$}};
                \filldraw[white] (2,1.4) circle (6pt);
                \node[text width=7pt] at (2,1.4) {\footnotesize{$x$}};
                \filldraw[white] (5.34,1.32) circle (6pt);
                \node[text width=7pt] at (5.34,1.32) {\footnotesize{$y$}};
            \end{tikzpicture}
        \end{subfigure}
        \caption{$b_0=1, \chi=2\pi, \rho=\sqrt{2}$, vykreslené souřadnice $\matu, x, y$} 
    \end{subfigure}
    \hfill
    \begin{subfigure}[b]{1\textwidth}
        \begin{subfigure}[b]{0.48\textwidth}
            \begin{tikzpicture}
                \node[inner sep=0pt, anchor=south west] (ds) at (0,0)
                {\adjincludegraphics[trim={{.14\width} {.15\height} {.15\width} {.22\height}}, width=1\textwidth, clip]{../img/kap03/flat_gyraton/null/null_skewed_gyraton_ring__r_1__mu_1__chi_2pi_vxy.pdf}};
                \filldraw[white] (0.39,2.91) circle (5pt);
                \node[text width=7pt] at (0.39,2.91) {\footnotesize{$\matv$}};
                \filldraw[white] (2.2,0.4) circle (6pt);
                \node[text width=7pt] at (2.2,0.4) {\footnotesize{$x$}};
                \filldraw[white] (5.87,1.14) circle (6pt);
                \node[text width=7pt] at (5.87,1.14) {\footnotesize{$y$}};
            \end{tikzpicture}
        \end{subfigure}
        \hfill
        \begin{subfigure}[b]{0.48\textwidth}
            \begin{tikzpicture}
                \node[inner sep=0pt, anchor=south west] (ds) at (0,0)
                {\adjincludegraphics[trim={{.12\width} {.10\height} {.15\width} {.18\height}}, width=1\textwidth, clip]{../img/kap03/flat_gyraton/null/null_skewed_top_gyraton_ring__r_1__mu_1__chi_2pi_vxy.pdf}};
                \filldraw[white] (0.39,3.4) circle (5pt);
                \node[text width=7pt] at (0.39,3.4) {\footnotesize{$\matv$}};
                \filldraw[white] (2,1.4) circle (6pt);
                \node[text width=7pt] at (2,1.4) {\footnotesize{$x$}};
                \filldraw[white] (5.34,1.32) circle (6pt);
                \node[text width=7pt] at (5.34,1.32) {\footnotesize{$y$}};
            \end{tikzpicture} 
        \end{subfigure}
        \caption{$b_0=1, \chi=2\pi, \rho=\sqrt{2}$, vykreslené souřadnice $\matv, x, y$}
    \end{subfigure}
    \caption{Nulové geodetiky procházející impulzní plochou Frolovova--Fursaevova gyratonu, pro lepší zřetelnost vykresleno z různých úhlů pohledu.}
    \label{fig:gyra_skewed_flat_null_b1_chi2}
\end{figure}

Časupodobné geodetiky byly vizualizovány v několika uspořádáních, nejprve v případě kdy impulzní
vlnoplochu protínají s tečným vektorem normovaným na jednotku
ve směru $\dot{\matu}=\frac{1}{2}, \dot{\matv} = 1, \dot{\eta}=0$. Tento případ je zobrazen na
obrázcích \ref{fig:gyra_flat_timelike_b1_chi1_2} a \ref{fig:gyra_flat_timelike_b1_chi2_vxy} a obdobně jako u
nulových geodetik pozorujeme vliv gyratonu reprezentovaný strháváním geodetiky ve smyslu rotace.

\begin{figure}[H]
    \centering
    \begin{subfigure}[b]{0.48\textwidth}
        \begin{tikzpicture}
            \node[inner sep=0pt, anchor=south west] (ds) at (0,0)
            {\adjincludegraphics[trim={{.17\width} {.2\height} {.17\width} {.22\height}}, width=1\textwidth, clip]{../img/kap03/flat_gyraton/timelike/gyraton_ring__r_2__mu_1__chi_1pi_uxy.pdf}};
            \filldraw[white] (0.45,2.8) circle (5pt);
            \node[text width=7pt] at (0.45,2.8) {\footnotesize{$\matu$}};
            \filldraw[white] (2,0.55) circle (6pt);
            \node[text width=7pt] at (2,0.55) {\footnotesize{$y$}};
            \filldraw[white] (5.68,0.88) circle (6pt);
            \node[text width=7pt] at (5.68,0.88) {\footnotesize{$x$}};
        \end{tikzpicture}
         \caption{$b_0=1, \chi=\pi, \rho=2\sqrt{2}$} 
    \end{subfigure}
    \begin{subfigure}[b]{0.48\textwidth}
        \begin{tikzpicture}
            \node[inner sep=0pt, anchor=south west] (ds) at (0,0)
            {\adjincludegraphics[trim={{.17\width} {.2\height} {.17\width} {.22\height}}, width=1\textwidth, clip]{../img/kap03/flat_gyraton/timelike/gyraton_ring__r_3__mu_1__chi_1pi_uxy.pdf}};
            \filldraw[white] (0.45,2.8) circle (5pt);
            \node[text width=7pt] at (0.45,2.8) {\footnotesize{$\matu$}};
            \filldraw[white] (2,0.55) circle (6pt);
            \node[text width=7pt] at (2,0.55) {\footnotesize{$y$}};
            \filldraw[white] (5.68,0.88) circle (6pt);
            \node[text width=7pt] at (5.68,0.88) {\footnotesize{$x$}};
        \end{tikzpicture}
         \caption{$b_0=1, \chi=\pi, \rho=3\sqrt{2}$} 
    \end{subfigure}
    \hfill
    \begin{subfigure}[b]{0.48\textwidth}
        \begin{tikzpicture}
            \node[inner sep=0pt, anchor=south west] (ds) at (0,0)
            {\adjincludegraphics[trim={{.17\width} {.2\height} {.17\width} {.22\height}},width=1\textwidth, clip]{../img/kap03/flat_gyraton/timelike/gyraton_ring__r_2__mu_1__chi_2pi_uxy.pdf}};
            \filldraw[white] (0.45,2.8) circle (5pt);
            \node[text width=7pt] at (0.45,2.8) {\footnotesize{$\matu$}};
            \filldraw[white] (2,0.55) circle (6pt);
            \node[text width=7pt] at (2,0.55) {\footnotesize{$y$}};
            \filldraw[white] (5.68,0.88) circle (6pt);
            \node[text width=7pt] at (5.68,0.88) {\footnotesize{$x$}};
        \end{tikzpicture}
         \caption{$b_0=1, \chi=2\pi, \rho=2\sqrt{2}$} 
    \end{subfigure}
    \begin{subfigure}[b]{0.48\textwidth}
        \begin{tikzpicture}
            \node[inner sep=0pt, anchor=south west] (ds) at (0,0)
            {\adjincludegraphics[trim={{.17\width} {.2\height} {.17\width} {.22\height}}, width=1\textwidth, clip]{../img/kap03/flat_gyraton/timelike/gyraton_ring__r_3__mu_1__chi_2pi_uxy.pdf}};
            \filldraw[white] (0.45,2.8) circle (5pt);
            \node[text width=7pt] at (0.45,2.8) {\footnotesize{$\matu$}};
            \filldraw[white] (2,0.55) circle (6pt);
            \node[text width=7pt] at (2,0.55) {\footnotesize{$y$}};
            \filldraw[white] (5.68,0.88) circle (6pt);
            \node[text width=7pt] at (5.68,0.88) {\footnotesize{$x$}};
        \end{tikzpicture}
         \caption{$b_0=1, \chi=2\pi, \rho=3\sqrt{2}$} 
    \end{subfigure}
    \caption{Časupodobné geodetiky procházející impulzní plochou gyratonu.}
    \label{fig:gyra_flat_timelike_b1_chi1_2}
\end{figure}

\begin{figure}[H]
    \centering
    \begin{subfigure}[b]{0.48\textwidth}
        \begin{tikzpicture}
            \node[inner sep=0pt, anchor=south west] (ds) at (0,0)
            {\adjincludegraphics[trim={{.17\width} {.2\height} {.17\width} {.22\height}}, width=1\textwidth, clip]{../img/kap03/flat_gyraton/timelike/gyraton_ring__r_1__mu_1__chi_2pi_vxy.pdf}};
            \filldraw[white] (0.45,2.8) circle (5pt);
            \node[text width=7pt] at (0.45,2.8) {\footnotesize{$\matv$}};
            \filldraw[white] (2,0.55) circle (6pt);
            \node[text width=7pt] at (2,0.55) {\footnotesize{$y$}};
            \filldraw[white] (5.68,0.88) circle (6pt);
            \node[text width=7pt] at (5.68,0.88) {\footnotesize{$x$}};
        \end{tikzpicture}
        \caption{$b_0=1, \chi=2\pi, \rho=\sqrt{2}$}
    \end{subfigure}
    \begin{subfigure}[b]{0.48\textwidth}
        \begin{tikzpicture}
            \node[inner sep=0pt, anchor=south west] (ds) at (0,0)
            {\adjincludegraphics[trim={{.17\width} {.2\height} {.17\width} {.22\height}}, width=1\textwidth, clip]{../img/kap03/flat_gyraton/timelike/gyraton_ring__r_2__mu_1__chi_2pi_vxy.pdf}};
            \filldraw[white] (0.45,2.8) circle (5pt);
            \node[text width=7pt] at (0.45,2.8) {\footnotesize{$\matv$}};
            \filldraw[white] (2,0.55) circle (6pt);
            \node[text width=7pt] at (2,0.55) {\footnotesize{$y$}};
            \filldraw[white] (5.68,0.88) circle (6pt);
            \node[text width=7pt] at (5.68,0.88) {\footnotesize{$x$}};
        \end{tikzpicture}
        \caption{$b_0=1, \chi=2\pi, \rho=2\sqrt{2}$} 
    \end{subfigure}
    \caption{Časupodobné geodetiky procházející impulzní plochou gyratonu. Částice před impulzem
    letí ve směru rostoucí souřadnice $\matv$, po impulzu jsou refraktovány zpět a složka čtyřrychlosti $\dot{\matv}$ vlivem refrakce přechází do záporných hodnot.}
    \label{fig:gyra_flat_timelike_b1_chi2_vxy}
\end{figure}

Ve vizualizaci na obrázcích \ref{fig:gyra_flat_timelike_b1_chi1_2_skewed} a \ref{fig:gyra_flat_timelike_b1_chi1_2_vxy_skewed} protínají geodetiky impulzní plochu s~jednotkovým tečným vektorem
ve směru $\dot{\matu} = \frac{1}{2}, \dot{\matv} = 1, \dot{\eta} = \frac{1}{2}$.

\begin{figure}[H]
    \centering
    \begin{subfigure}[b]{0.48\textwidth}
        \begin{tikzpicture}
            \node[inner sep=0pt, anchor=south west] (ds) at (0,0)
            {\adjincludegraphics[trim={{.1\width} {.13\height} {.12\width} {.22\height}}, width=1\textwidth, clip]{../img/kap03/flat_gyraton/timelike/skewed_gyraton_ring__r_1__mu_1__chi_1pi_uxy.pdf}};
            \filldraw[white] (0.47,2.6) circle (5pt);
            \node[text width=7pt] at (0.47,2.6) {\footnotesize{$\matu$}};
            \filldraw[white] (2.56,0.35) circle (6pt);
            \node[text width=7pt] at (2.56,0.35) {\footnotesize{$x$}};
            \filldraw[white] (5.82,1.1) circle (6pt);
            \node[text width=7pt] at (5.82,1.1) {\footnotesize{$y$}};
        \end{tikzpicture}
         \caption{$b_0=1, \chi=\pi, \rho=\sqrt{2}$} 
    \end{subfigure}
    \begin{subfigure}[b]{0.48\textwidth}
        \begin{tikzpicture}
            \node[inner sep=0pt, anchor=south west] (ds) at (0,0)
            {\adjincludegraphics[trim={{.1\width} {.13\height} {.12\width} {.22\height}}, width=1\textwidth, clip]{../img/kap03/flat_gyraton/timelike/skewed_gyraton_ring__r_2__mu_1__chi_1pi_uxy.pdf}};
            \filldraw[white] (0.47,2.6) circle (5pt);
            \node[text width=7pt] at (0.47,2.6) {\footnotesize{$\matu$}};
            \filldraw[white] (2.54,0.3) circle (5pt);
            \node[text width=7pt] at (2.54,0.3) {\footnotesize{$x$}};
            \filldraw[white] (5.82,1.1) circle (6pt);
            \node[text width=7pt] at (5.82,1.1) {\footnotesize{$y$}};
        \end{tikzpicture}
         \caption{$b_0=1, \chi=\pi, \rho=2\sqrt{2}$} 
    \end{subfigure}
    \hfill
    \begin{subfigure}[b]{0.48\textwidth}
        \begin{tikzpicture}
            \node[inner sep=0pt, anchor=south west] (ds) at (0,0)
            {\adjincludegraphics[trim={{.1\width} {.13\height} {.12\width} {.22\height}},width=1\textwidth, clip]{../img/kap03/flat_gyraton/timelike/skewed_gyraton_ring__r_1__mu_1__chi_2pi_uxy.pdf}};
            \filldraw[white] (0.47,2.6) circle (5pt);
            \node[text width=7pt] at (0.47,2.6) {\footnotesize{$\matu$}};
            \filldraw[white] (2.56,0.33) circle (5pt);
            \node[text width=7pt] at (2.56,0.33) {\footnotesize{$x$}};
            \filldraw[white] (5.82,1.1) circle (6pt);
            \node[text width=7pt] at (5.82,1.1) {\footnotesize{$y$}};
        \end{tikzpicture}
         \caption{$b_0=1, \chi=2\pi, \rho=\sqrt{2}$} 
    \end{subfigure}
    \begin{subfigure}[b]{0.48\textwidth}
        \begin{tikzpicture}
            \node[inner sep=0pt, anchor=south west] (ds) at (0,0)
            {\adjincludegraphics[trim={{.1\width} {.13\height} {.12\width} {.22\height}}, width=1\textwidth, clip]{../img/kap03/flat_gyraton/timelike/skewed_gyraton_ring__r_2__mu_1__chi_2pi_uxy.pdf}};
            \filldraw[white] (0.47,2.6) circle (5pt);
            \node[text width=7pt] at (0.47,2.6) {\footnotesize{$\matu$}};
            \filldraw[white] (2.56,0.35) circle (6pt);
            \node[text width=7pt] at (2.56,0.35) {\footnotesize{$x$}};
            \filldraw[white] (5.82,1.1) circle (6pt);
            \node[text width=7pt] at (5.82,1.1) {\footnotesize{$y$}};
        \end{tikzpicture}
         \caption{$b_0=1, \chi=2\pi, \rho=2\sqrt{2}$} 
    \end{subfigure}
    \caption{Časupodobné geodetiky procházející impulzní plochou gyratonu.}
    \label{fig:gyra_flat_timelike_b1_chi1_2_skewed}
\end{figure}

\begin{figure}[H]
    \centering
    \begin{subfigure}[b]{0.48\textwidth}
        \begin{tikzpicture}
            \node[inner sep=0pt, anchor=south west] (ds) at (0,0)
            {\adjincludegraphics[trim={{.1\width} {.13\height} {.12\width} {.22\height}}, width=1\textwidth, clip]{../img/kap03/flat_gyraton/timelike/skewed_gyraton_ring__r_1__mu_1__chi_2pi_vxy.pdf}};
            \filldraw[white] (0.47,2.6) circle (5pt);
            \node[text width=7pt] at (0.47,2.6) {\footnotesize{$\matv$}};
            \filldraw[white] (2.56,0.35) circle (6pt);
            \node[text width=7pt] at (2.56,0.35) {\footnotesize{$x$}};
            \filldraw[white] (5.82,1.1) circle (6pt);
            \node[text width=7pt] at (5.82,1.1) {\footnotesize{$y$}};
        \end{tikzpicture}
        \caption{$b_0=1, \chi=2\pi, \rho=\sqrt{2}$}
    \end{subfigure}
    \begin{subfigure}[b]{0.48\textwidth}
        \begin{tikzpicture}
            \node[inner sep=0pt, anchor=south west] (ds) at (0,0)
            {\adjincludegraphics[trim={{.1\width} {.13\height} {.12\width} {.22\height}}, width=1\textwidth, clip]{../img/kap03/flat_gyraton/timelike/skewed_gyraton_ring__r_2__mu_1__chi_2pi_vxy.pdf}};
            \filldraw[white] (0.47,2.6) circle (5pt);
            \node[text width=7pt] at (0.47,2.6) {\footnotesize{$\matv$}};
            \filldraw[white] (2.56,0.3) circle (5pt);
            \node[text width=7pt] at (2.56,0.3) {\footnotesize{$x$}};
            \filldraw[white] (5.82,1.1) circle (6pt);
            \node[text width=7pt] at (5.82,1.1) {\footnotesize{$y$}};
        \end{tikzpicture}
        \caption{$b_0=1, \chi=2\pi, \rho=2\sqrt{2}$} 
    \end{subfigure}
    \caption{Časupodobné geodetiky procházející impulzní plochou gyratonu, vykreslené souřadnice $\matv, x, y$.}
    \label{fig:gyra_flat_timelike_b1_chi1_2_vxy_skewed}
\end{figure}

\subsection{Geodetický pohyb v přítomnosti gyratonu~s profilem odpovídajícím Hottově--Tanakově řešení}
Dále byly zkoumány geodetiky v explicitní realizaci gyratonového prostoročasu s parametrem $J = \frac{\chi}{2 i \eta} \Theta(\matu), \Lambda \not= 0$, s profilovou funkcí $H$ odpovídající
Hottově--Tanakově řešení \eqref{eq:Hotta_Tanaka_H_conf_flat_coords}.
Stejně jako v plochém případě je gyratonový efekt prostoročasu za vlnou zřetelný už na skoku v úhlové části čtyřrychlosti.

Na obrázcích \ref{fig:gyra_ht_null_b1_chi12_lmb1} a \ref{fig:gyra_ht_null_b1_chi12_lmb-1} je vizualizace nulových geodetik na de Sitterově a anti-de Sitterově pozadí v axiálně symetrickém uspořádání.
Geodetiky procházejí, stejně jako v případě $\Lambda = 0$, v různých vzdálenostech od osy symetrie s tečným vektorem při průchodu $\dot\matu = 1, \dot\matv=0, \dot\eta=0$.
K efektu způsobenému impulzní vlnou, který strhne geodetiky směrem k ose symetrie (respektive směrem od, pro opačné znaménko parametru $b_0$ v případě Hottova--Tanakova řešení), se přidává efekt gyratonu,
který geodetiky strhne v rotačním smyslu.

\begin{figure}[H]
    \centering
    \begin{subfigure}[b]{0.45\textwidth}
        \begin{tikzpicture}
            \node[inner sep=0pt, anchor=south west] (ds) at (0,0)
            {\adjincludegraphics[trim={{.12\width} {.10\height} {.16\width} {.22\height}}, width=1\textwidth, clip]{../img/kap03/ht_gyraton/null/null__udot_1__vdot_0__ring__r_1__mu_1__chi_1pi_uxy.pdf}};
            \filldraw[white] (0.31,2.81) circle (4pt);
            \node[text width=7pt] at (0.31,2.81) {\footnotesize{$\matu$}};
            \filldraw[white] (2.35,0.5) circle (4pt);
            \node[text width=7pt] at (2.35,0.5) {\footnotesize{$x$}};
            \filldraw[white] (5.68,1.48) circle (5pt);
            \node[text width=7pt] at (5.68,1.48) {\footnotesize{$y$}};
        \end{tikzpicture}
        \caption{$b_0=1, \chi=\pi, \rho=\sqrt{2}, \Lambda = 1$} 
    \end{subfigure}
    \hfill
    \begin{subfigure}[b]{0.45\textwidth}
        \begin{tikzpicture}
            \node[inner sep=0pt, anchor=south west] (ds) at (0,0)
            {\adjincludegraphics[trim={{.12\width} {.10\height} {.16\width} {.22\height}}, width=1\textwidth, clip]{../img/kap03/ht_gyraton/null/null__udot_1__vdot_0__ring__r_2__mu_1__chi_1pi_uxy.pdf}};
            \filldraw[white] (0.31,2.81) circle (4pt);
            \node[text width=7pt] at (0.31,2.81) {\footnotesize{$\matu$}};
            \filldraw[white] (2.35,0.5) circle (4pt);
            \node[text width=7pt] at (2.35,0.5) {\footnotesize{$x$}};
            \filldraw[white] (5.68,1.48) circle (5pt);
            \node[text width=7pt] at (5.68,1.48) {\footnotesize{$y$}};
        \end{tikzpicture}
        \caption{$b_0=1, \chi=\pi, \rho=2\sqrt{2}, \Lambda = 1$}
    \end{subfigure}
    \hfill
    \begin{subfigure}[b]{0.45\textwidth}
        \begin{tikzpicture}
            \node[inner sep=0pt, anchor=south west] (ds) at (0,0)
            {\adjincludegraphics[trim={{.12\width} {.10\height} {.15\width} {.22\height}}, width=1\textwidth, clip]{../img/kap03/ht_gyraton/null/null__udot_1__vdot_0__ring__r_1__mu_1__chi_2pi_uxy.pdf}};
            \filldraw[white] (0.31,2.76) circle (4pt);
            \node[text width=7pt] at (0.31,2.76) {\footnotesize{$\matu$}};
            \filldraw[white] (2.35,0.5) circle (4pt);
            \node[text width=7pt] at (2.35,0.5) {\footnotesize{$x$}};
            \filldraw[white] (5.62,1.44) circle (5pt);
            \node[text width=7pt] at (5.62,1.44) {\footnotesize{$y$}};
        \end{tikzpicture}
        \caption{$b_0=1, \chi=2\pi, \rho=\sqrt{2}, \Lambda = 1$}
    \end{subfigure}
    \hfill
    \begin{subfigure}[b]{0.45\textwidth}
        \begin{tikzpicture}
            \node[inner sep=0pt, anchor=south west] (ds) at (0,0)
            {\adjincludegraphics[trim={{.12\width} {.10\height} {.15\width} {.22\height}}, width=1\textwidth, clip]{../img/kap03/ht_gyraton/null/null__udot_1__vdot_0__ring__r_2__mu_1__chi_2pi_uxy.pdf}};
            \filldraw[white] (0.31,2.76) circle (4pt);
            \node[text width=7pt] at (0.31,2.76) {\footnotesize{$\matu$}};
            \filldraw[white] (2.35,0.5) circle (4pt);
            \node[text width=7pt] at (2.35,0.5) {\footnotesize{$x$}};
            \filldraw[white] (5.62,1.44) circle (5pt);
            \node[text width=7pt] at (5.62,1.44) {\footnotesize{$y$}};
        \end{tikzpicture} 
        \caption{$b_0=1, \chi=2\pi, \rho=2\sqrt{2}, \Lambda = 1$}
    \end{subfigure}
    \caption{Nulové geodetiky procházející impulzní plochou "gyratonového Hottova--Tanakova řešení"\ s čtyřrychlostí $\dot\matu = 1, \dot\matv=0, \dot\eta=0$. Případ de Sitterova pozadí.}
    \label{fig:gyra_ht_null_b1_chi12_lmb1}
\end{figure}

\begin{figure}[H]
    \centering
    \begin{subfigure}[b]{0.48\textwidth}
        \begin{tikzpicture}
            \node[inner sep=0pt, anchor=south west] (ds) at (0,0)
            {\adjincludegraphics[trim={{.12\width} {.10\height} {.16\width} {.22\height}}, width=1\textwidth, clip]{../img/kap03/ht_gyraton/null/null__udot_1__vdot_0__ring__r_1__lambda-1__mu_1__chi_1pi_uxy.pdf}};
            \filldraw[white] (0.34,3) circle (4pt);
            \node[text width=7pt] at (0.34,3) {\footnotesize{$\matu$}};
            \filldraw[white] (2.54,0.52) circle (4pt);
            \node[text width=7pt] at (2.54,0.52) {\footnotesize{$x$}};
            \filldraw[white] (6,1.54) circle (5pt);
            \node[text width=7pt] at (6,1.54) {\footnotesize{$y$}};
        \end{tikzpicture}
        \caption{$b_0=1, \chi=\pi, \rho=\sqrt{2}, \Lambda = -1$} 
    \end{subfigure}
    \hfill
    \begin{subfigure}[b]{0.48\textwidth}
        \begin{tikzpicture}
            \node[inner sep=0pt, anchor=south west] (ds) at (0,0)
            {\adjincludegraphics[trim={{.12\width} {.10\height} {.16\width} {.22\height}}, width=1\textwidth, clip]{../img/kap03/ht_gyraton/null/null__udot_1__vdot_0__ring__r_2__lambda-1__mu_1__chi_1pi_uxy.pdf}};
            \filldraw[white] (0.34,3) circle (5pt);
            \node[text width=7pt] at (0.34,3) {\footnotesize{$\matu$}};
            \filldraw[white] (2.54,0.52) circle (4pt);
            \node[text width=7pt] at (2.54,0.52) {\footnotesize{$x$}};
            \filldraw[white] (6,1.54) circle (5pt);
            \node[text width=7pt] at (6,1.54) {\footnotesize{$y$}};
        \end{tikzpicture}
        \caption{$b_0=1, \chi=\pi, \rho=2\sqrt{2}, \Lambda = -1$}
    \end{subfigure}
    \hfill
    \begin{subfigure}[b]{0.48\textwidth}
        \begin{tikzpicture}
            \node[inner sep=0pt, anchor=south west] (ds) at (0,0)
            {\adjincludegraphics[trim={{.12\width} {.10\height} {.16\width} {.22\height}}, width=1\textwidth, clip]{../img/kap03/ht_gyraton/null/null__udot_1__vdot_0__ring__r_1__lambda-1__mu_1__chi_2pi_uxy.pdf}};
            \filldraw[white] (0.34,3) circle (5pt);
            \node[text width=7pt] at (0.34,3) {\footnotesize{$\matu$}};
            \filldraw[white] (2.54,0.52) circle (4pt);
            \node[text width=7pt] at (2.54,0.52) {\footnotesize{$x$}};
            \filldraw[white] (6,1.54) circle (5pt);
            \node[text width=7pt] at (6,1.54) {\footnotesize{$y$}};
        \end{tikzpicture}
        \caption{$b_0=1, \chi=2\pi, \rho=\sqrt{2}, \Lambda = -1$}
    \end{subfigure}
    \hfill
    \begin{subfigure}[b]{0.48\textwidth}
        \begin{tikzpicture}
            \node[inner sep=0pt, anchor=south west] (ds) at (0,0)
            {\adjincludegraphics[trim={{.12\width} {.10\height} {.16\width} {.22\height}}, width=1\textwidth, clip]{../img/kap03/ht_gyraton/null/null__udot_1__vdot_0__ring__r_2__lambda-1__mu_1__chi_2pi_uxy.pdf}};
            \filldraw[white] (0.34,3) circle (5pt);
            \node[text width=7pt] at (0.34,3) {\footnotesize{$\matu$}};
            \filldraw[white] (2.54,0.52) circle (4pt);
            \node[text width=7pt] at (2.54,0.52) {\footnotesize{$x$}};
            \filldraw[white] (6,1.54) circle (5pt);
            \node[text width=7pt] at (6,1.54) {\footnotesize{$y$}};
        \end{tikzpicture} 
        \caption{$b_0=1, \chi=2\pi, \rho=2\sqrt{2}, \Lambda = -1$}
    \end{subfigure}
    \caption{Nulové geodetiky procházející impulzní plochou "gyratonového Hottova--Tanakova řešení"\ pro anti-de Sitterovo pozadí.}
    \label{fig:gyra_ht_null_b1_chi12_lmb-1}
\end{figure}

Jako poslední byly vizualizovány časupodobné geodetiky procházející impulzní nadplochou s čtyřrychlostí (normovanou na -1) ve směru $\dot \matu = \frac{1}{2},~ \dot \matv = 1, \dot \eta = 0$ (viz obrázek \ref{fig:gyra_ht_timelike_b1_chi1}) a dále, pro úplnost, časupodobné
geodetiky procházející $\matu = 0$ s čtyřrychlostí ve směru $\dot \matu = \frac{1}{2}, \dot \matv = 1, \dot \eta = \frac{3}{10}$ (viz obrázek \ref{fig:skewed_gyra_ht_timelike_b1_chi1}).

\begin{figure}[H]
    \centering
    \begin{subfigure}[b]{0.48\textwidth}
        \begin{tikzpicture}
            \node[inner sep=0pt, anchor=south west] (ds) at (0,0)
            {\adjincludegraphics[trim={{.12\width} {.10\height} {.16\width} {.22\height}}, width=1\textwidth, clip]{../img/kap03/ht_gyraton/timelike/timelike__udot_1__vdot_0__ring__r_1__lambda1__mu_1__chi_1pi_uxy.pdf}};
            \filldraw[white] (0.34,3) circle (5pt);
            \node[text width=7pt] at (0.34,3) {\footnotesize{$\matu$}};
            \filldraw[white] (2.54,0.52) circle (4pt);
            \node[text width=7pt] at (2.54,0.52) {\footnotesize{$x$}};
            \filldraw[white] (6,1.54) circle (5pt);
            \node[text width=7pt] at (6,1.54) {\footnotesize{$y$}};
        \end{tikzpicture}
        \caption{$b_0=1, \chi=\pi, \rho=\sqrt{2}, \Lambda = 1$}
    \end{subfigure}
    \hfill
    \begin{subfigure}[b]{0.48\textwidth}
        \begin{tikzpicture}
            \node[inner sep=0pt, anchor=south west] (ds) at (0,0)
            {\adjincludegraphics[trim={{.12\width} {.10\height} {.16\width} {.22\height}}, width=1\textwidth, clip]{../img/kap03/ht_gyraton/timelike/timelike__udot_1__vdot_0__ring__r_2__lambda1__mu_1__chi_1pi_uxy.pdf}};
            \filldraw[white] (0.34,3) circle (5pt);
            \node[text width=7pt] at (0.34,3) {\footnotesize{$\matu$}};
            \filldraw[white] (2.54,0.52) circle (4pt);
            \node[text width=7pt] at (2.54,0.52) {\footnotesize{$x$}};
            \filldraw[white] (6,1.54) circle (5pt);
            \node[text width=7pt] at (6,1.54) {\footnotesize{$y$}};
        \end{tikzpicture}
        \caption{$b_0=1, \chi=\pi, \rho=2\sqrt{2}, \Lambda = 1$}
    \end{subfigure}
    \hfill
    \begin{subfigure}[b]{0.48\textwidth}
        \begin{tikzpicture}
            \node[inner sep=0pt, anchor=south west] (ds) at (0,0)
            {\adjincludegraphics[trim={{.12\width} {.10\height} {.16\width} {.22\height}}, width=1\textwidth, clip]{../img/kap03/ht_gyraton/timelike/timelike__udot_1__vdot_0__ring__r_1__lambda-1__mu_1__chi_1pi_uxy.pdf}};
            \filldraw[white] (0.34,3) circle (5pt);
            \node[text width=7pt] at (0.34,3) {\footnotesize{$\matu$}};
            \filldraw[white] (2.54,0.52) circle (4pt);
            \node[text width=7pt] at (2.54,0.52) {\footnotesize{$x$}};
            \filldraw[white] (6,1.54) circle (5pt);
            \node[text width=7pt] at (6,1.54) {\footnotesize{$y$}};
        \end{tikzpicture}
        \caption{$b_0=1, \chi=\pi, \rho=\sqrt{2}, \Lambda = -1$}
    \end{subfigure}
    \hfill
    \begin{subfigure}[b]{0.48\textwidth}
        \begin{tikzpicture}
            \node[inner sep=0pt, anchor=south west] (ds) at (0,0)
            {\adjincludegraphics[trim={{.12\width} {.10\height} {.15\width} {.18\height}}, width=1\textwidth, clip]{../img/kap03/ht_gyraton/timelike/timelike__udot_1__vdot_0__ring__r_2__lambda-1__mu_1__chi_1pi_uxy.pdf}};
            \filldraw[white] (0.34,2.98) circle (5pt);
            \node[text width=7pt] at (0.34,2.98) {\footnotesize{$\matu$}};
            \filldraw[white] (2.54,0.52) circle (4pt);
            \node[text width=7pt] at (2.54,0.52) {\footnotesize{$x$}};
            \filldraw[white] (6,1.54) circle (5pt);
            \node[text width=7pt] at (6,1.54) {\footnotesize{$y$}};
        \end{tikzpicture} 
        \caption{$b_0=1, \chi=\pi, \rho=2\sqrt{2}, \Lambda = -1$}
    \end{subfigure}
    \caption{Časupodobné geodetiky procházející impulzní plochou "gyratonového Hottova--Tanakova řešení"\ ve směru $\dot \matu = 0,5,~ \dot \matv = 1, \dot \eta = 0$ pro (anti-)de Sitterovo pozadí.}
    \label{fig:gyra_ht_timelike_b1_chi1}
\end{figure}

\begin{figure}[H]
    \centering
    \begin{subfigure}[b]{0.48\textwidth}
        \begin{tikzpicture}
            \node[inner sep=0pt, anchor=south west] (ds) at (0,0)
            {\adjincludegraphics[trim={{.12\width} {.10\height} {.15\width} {.22\height}}, width=1\textwidth, clip]{../img/kap03/ht_gyraton/timelike/skewed_timelike__udot_1__vdot_0__ring__r_1__lambda1__mu_1__chi_1pi_uxy.pdf}};
            \filldraw[white] (0.34,2.95) circle (5pt);
            \node[text width=7pt] at (0.34,2.95) {\footnotesize{$\matu$}};
            \filldraw[white] (2.54,0.52) circle (4pt);
            \node[text width=7pt] at (2.54,0.52) {\footnotesize{$x$}};
            \filldraw[white] (6,1.54) circle (5pt);
            \node[text width=7pt] at (6,1.54) {\footnotesize{$y$}};
        \end{tikzpicture}
        \caption{$b_0=1, \chi=\pi, \rho=\sqrt{2}, \Lambda = 1$}
    \end{subfigure}
    \hfill
    \begin{subfigure}[b]{0.48\textwidth}
        \begin{tikzpicture}
            \node[inner sep=0pt, anchor=south west] (ds) at (0,0)
            {\adjincludegraphics[trim={{.12\width} {.10\height} {.15\width} {.22\height}}, width=1\textwidth, clip]{../img/kap03/ht_gyraton/timelike/skewed_timelike__udot_1__vdot_0__ring__r_1__lambda-1__mu_1__chi_1pi_uxy.pdf}};
            \filldraw[white] (0.34,2.95) circle (5pt);
            \node[text width=7pt] at (0.34,2.95) {\footnotesize{$\matu$}};
            \filldraw[white] (2.54,0.52) circle (4pt);
            \node[text width=7pt] at (2.54,0.52) {\footnotesize{$x$}};
            \filldraw[white] (6,1.54) circle (5pt);
            \node[text width=7pt] at (6,1.54) {\footnotesize{$y$}};
        \end{tikzpicture} 
        \caption{$b_0=1, \chi=\pi, \rho=\sqrt{2}, \Lambda = -1$}
    \end{subfigure}
    \caption{Časupodobné geodetiky procházející impulzní plochou "gyratonového Hottova--Tanakova řešení"\ ve směru $\dot \matu = \frac{1}{2}, \dot \matv = 1, \dot \eta = 0.3$ pro (anti-)de Sitterovo pozadí.}
    \label{fig:skewed_gyra_ht_timelike_b1_chi1}
\end{figure}

\cleardoublepage