\chapter{Neexpandující gravitační vlny s gyratonovými členy}
Distribuční vyjádření metriky impulzní vlny \ref{eq:nonexp_distr_metric_omega} není nejobecnější metrikou popisující
impulzní vlny. Můžeme totiž uvažovat i tzv. mimodiagonální členy, se kterými pak metrika nabývá tvaru
\begin{equation}
    \label{eq:nonexp_gyra_distrib_metric_omega}
    \begin{split}
        \mathrm{d}s^2=&\frac{2\mathrm{d}\eta~\mathrm{d}\bar{\eta} - 2 \mathrm{d}\mathcal{U}~\mathrm{d}\mathcal{V} + 2H(\eta, \bar{\eta}) \delta(\mathcal{U}) 
        ~\mathrm{d}\mathcal{U}^2}{\left[1+\frac{1}{6}\Lambda(\eta \bar{\eta}-\mathcal{U}\mathcal{V})\right]^2} \\
        &+ \frac{2J\left(\eta, \bar{\eta}, \mathcal{U}\right) \mathrm{d}\eta~\mathrm{d}\mathcal{U}
        +2\overline{J}\left(\eta, \bar{\eta}, \mathcal{U}\right) \mathrm{d}\bar{\eta}~\mathrm{d}\mathcal{U}}{\left[1+\frac{1}{6}\Lambda(\eta \bar{\eta}-\mathcal{U}\mathcal{V})\right]^2}.
    \end{split}
\end{equation}
Obvyklým postupem je odstranění členů s funkcí $J$ vhodnou souřadnicovou transformací, to ale vede k odstranění
možného rotačního charakteru zdroje gravitační vlny, taková transformace pak není globální, dochází k
zanedbání topologických vlastností celého prostoročasu.

\section{Konstrukce}
...
\subsection{"Cut and paste"\ metoda konstrukce}
...
\subsection{Spojitý tvar metriky}
...
\section{Interakce s testovacími čísticemi v prostoročasech s gyratony}
\subsection{Refrakční rovnice}
\textcolor{red}{Stejná struktura jako předchozí kapitola, stačí přepsat a trochu okomentovat
(např. "delty" ve skocích v čtyřrychlosti - započítat metrické členy)}