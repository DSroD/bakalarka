\chapter{Neexpandující gravitační vlny s gyratonovými členy}
Distribuční vyjádření metriky impulzní vlny \eqref{eq:nonexp_distr_metric_omega} není nejobecnější metrikou popisující
impulzní vlny. Už Brinkmann \textcolor{red}{(nezapomenout Brinkmanna v úvodu BP!)} uvažoval i tzv. mimodiagonální členy, se kterými pak (i v případě nenulové kosmologické konstanty)
metrika nabývá tvaru
\begin{equation}
    \label{eq:nonexp_gyra_distrib_metric_omega}
    \begin{split}
        \mathrm{d}s^2=&\frac{2\mathrm{d}\eta~\mathrm{d}\bar{\eta} - 2 \mathrm{d}\mathcal{U}~\mathrm{d}\mathcal{V} + 2H(\eta, \bar{\eta}) \delta(\mathcal{U}) 
        ~\mathrm{d}\mathcal{U}^2}{\left[1+\frac{1}{6}\Lambda(\eta \bar{\eta}-\mathcal{U}\mathcal{V})\right]^2} \\
        &+ \frac{2J\left(\eta, \bar{\eta}, \mathcal{U}\right) \mathrm{d}\eta~\mathrm{d}\mathcal{U}
        +2\overline{J}\left(\eta, \bar{\eta}, \mathcal{U}\right) \mathrm{d}\bar{\eta}~\mathrm{d}\mathcal{U}}{\left[1+\frac{1}{6}\Lambda(\eta \bar{\eta}-\mathcal{U}\mathcal{V})\right]^2}.
    \end{split}
\end{equation}
Obvyklým postupem je odstranění členů s funkcí $J$ vhodnou souřadnicovou transformací, to ale vede k odstranění
možného rotačního charakteru zdroje gravitační vlny, taková transformace pak není globální, dochází k
zanedbání topologických vlastností celého prostoročasu.

\section{Konstrukce}
\textcolor{red}{Má tahle sekce vůbec smysl? Možná přejmenovat na zobecnění na prostoročasy s gyratonovými členy a 
totožnost cut and paste nechat jen v "úvodním slově".}
Konstrukce neexpandujících gracitačních vln s gyratonovými členy je obdobná případu bez gyratonových
členů. Opět můžeme využít Penroseovu "cut and paste"\ metodu, v případě gyratonických prostoročasů ale bude
konstrukce totožná s konstrukcí popsanou v \eqref{sec:cut_and_paste_konstrukce1} a jak je ukázáno v článku \cite{Podolsky_2017}, Penroseovy lepící podmínky 
v přítomnosti gyratonů nabývají tvaru \eqref{eq:lepici_podminky}. \textcolor{red}{(přeformulovat něják lépe...)},

\subsection{"Cut and paste"\ metoda konstrukce}
Metoda "cut and paste"\ ... \textcolor{red}{Tohle asi pryč.}
\subsection{Spojitý tvar metriky}
Tvar spojité metriky pro gyratonové neexpandující impulsní vlny obdržíme zobecněním transformace \eqref{eq:nonexp_cont_full_transform},
do tvaru který nalezli Podolský, Švarc, Säman a Steinbauer v \cite{Podolsky_2017}
\begin{equation}
    \label{eq:gyraton_cont_transformation}
    \begin{split}
        \matu &= U \\
        \matv &= V + \Theta H + U_{+} H_{,Z} H_{,\bar{Z}} + W \\
        \eta &= \left(Z + U_{+} H_{,\bar{Z}}\right) \exp \left(i F \right),
    \end{split}
\end{equation}

kde opět $H = H(Z, \bar{Z})$, zároveň funkce $W = W(Z, \bar{Z}, U)$ a $F = F(Z, \bar{Z}, U)$ jsou reálné a splňují
\begin{equation}
    \begin{split}
        F_{,U} &= \frac{i\bar{J}}{Z + U_{+}H_{,\bar{Z}}} \exp{\left(-iF\right)}, \\
        W_{,U} &= -J \bar{J}.
    \end{split}
\end{equation}
Zavedením
\begin{equation}
    \zeta \equiv Z + U_{+} H_{,\bar{Z}}
\end{equation}
a substitucí \eqref{eq:gyraton_cont_transformation} do \eqref{eq:nonexp_gyra_distrib_metric_omega}


\section{Interakce s testovacími čísticemi v prostoročasech s gyratony}
\subsection{Refrakční rovnice}
\textcolor{red}{Stejná struktura jako předchozí kapitola, stačí přepsat a trochu okomentovat
(např. "delty" ve skocích v čtyřrychlosti - započítat metrické členy)}