\chapter{Neexpandující gravitační vlny s gyratonovými členy}
Distribuční vyjádření metriky impulsní vlny \eqref{eq:nonexp_distr_metric_omega} není nejobecnější metrikou popisující
impulsní vlny. Lze jí rozšířit o mimodiagonální členy do tvaru
\begin{equation}
    \label{eq:nonexp_gyra_distrib_metric_omega}
    \begin{split}
        \mathrm{d}s^2=&\frac{2\mathrm{d}\eta~\mathrm{d}\bar{\eta} - 2 \mathrm{d}\mathcal{U}~\mathrm{d}\mathcal{V} + 2H(\eta, \bar{\eta}) \delta(\mathcal{U}) 
        ~\mathrm{d}\mathcal{U}^2}{\left[1+\frac{1}{6}\Lambda(\eta \bar{\eta}-\mathcal{U}\mathcal{V})\right]^2} \\
        &+ \frac{2J\left(\eta, \bar{\eta}, \mathcal{U}\right) \mathrm{d}\eta~\mathrm{d}\mathcal{U}
        +2\overline{J}\left(\eta, \bar{\eta}, \mathcal{U}\right) \mathrm{d}\bar{\eta}~\mathrm{d}\mathcal{U}}{\left[1+\frac{1}{6}\Lambda(\eta \bar{\eta}-\mathcal{U}\mathcal{V})\right]^2}.
    \end{split}
\end{equation}
Tvar metriky s mimodiagonálními členy odpovídající gravitační vlně byl uvažován už v Brinkmannově studii \cite{Brinkmann1925} Einsteinových prostoročasů svázaných konformními transformacemi.

Obvyklým postupem je odstranění členů s funkcí $J$ vhodnou souřadnicovou transformací (Brinkmannova forma metriky pp-vln \textcolor{red}{tady bude reference} se také běžně uvádí už bez mimodiagonálních členů), to ale vede k odstranění
možného rotačního charakteru zdroje gravitační vlny, taková transformace pak není globální a dochází ke změně topologických vlastností celého prostoročasu.
V této kapitole budeme studovat a vizualizovat vliv nediagonálních prvků metriky.

\section{Zobecnění impulsních vln na prostoročasy s gyratonovými členy}
Konstrukce neexpandujících gracitačních vln s mimodagonálními gyratonovými členy je obdobná konstrukci prostoročasů bez gyratonů z minulé kapitoly, popíšeme ji formalismem spojitých souřadnic
použitým v \cite{Podolsky_2017}, kde jsou také odvozeny refrakční rovnice, pomocí kterých budeme vizualizovat geodetický pohyb procházející impulsní vlnoplochou.

Penroseova "cut and paste"\ konstrukce vede i v případě gyratonických prostoročasů na Penroseovy lepící podmínky ve tvaru \eqref{eq:lepici_podminky}, začneme tedy rovnou zobecněním
spojitého tvaru metriky z předchozí kapitoly.

\subsection{Zobecnění spojitého tvaru metriky}
Tvar spojité metriky pro gyratonové neexpandující impulsní vlny obdržíme zobecněním transformace \eqref{eq:nonexp_cont_full_transform},
do tvaru který nalezli Podolský, Švarc, Säman a Steinbauer v \cite{Podolsky_2017}
\begin{equation}
    \label{eq:gyraton_cont_transformation}
    \begin{split}
        \matu &= U \\
        \matv &= V + \Theta H + U_{+} H_{,Z} H_{,\bar{Z}} + W \\
        \eta &= \left(Z + U_{+} H_{,\bar{Z}}\right) \exp \left(i F \right),
    \end{split}
\end{equation}

kde opět $H = H(Z, \bar{Z})$, zároveň funkce $W = W(Z, \bar{Z}, U)$ a $F = F(Z, \bar{Z}, U)$ jsou reálné a splňují
\begin{equation}
    \label{eq:podminky_F_a_W}
    \begin{split}
        F_{,U} &= \frac{i\bar{J}}{Z + U_{+}H_{,\bar{Z}}} \exp{\left(-iF\right)}, \\
        W_{,U} &= -J \bar{J}, \\
        W &= 0 \text{ pro } U \leq 0.
    \end{split}
\end{equation}
Zavedením
\begin{equation}
    \zeta \equiv Z + U_{+} H_{,\bar{Z}}
\end{equation}
a prostorového diferenciálu $\underline{\rmd}$ tak, že
\begin{equation}
    \label{eq:spatial_differential_on_functions}
    \begin{split}
        \underline{\rmd}\zeta &\equiv \rmd Z + U_{+}(H_{,\bar{Z}Z}\rmd Z + H_{,\bar{Z}\bar{Z}}\rmd \bar{Z}), \\
        \underline{\rmd}H &\equiv H_{,Z}\rmd Z + H_{,\bar{Z}}\rmd \bar{Z}, \\
        \underline{\rmd}F &\equiv F_{,Z}\rmd Z + F_{,\bar{Z}}\rmd \bar{Z}, \\
        \underline{\rmd}W &\equiv W_{,Z}\rmd Z + W_{,\bar{Z}}\rmd \bar{Z}.
    \end{split}
\end{equation}
Dosazením \eqref{eq:gyraton_cont_transformation} do \eqref{eq:nonexp_gyra_distrib_metric_omega} a využitím vztahů \eqref{eq:podminky_F_a_W},
\eqref{eq:spatial_differential_on_functions} a multiplikativních pravidel z nelineární teorie distribucí,
\begin{equation}
    \Theta^2 = \Theta, ~~~~~~ \Theta U_{+} = U_{+},
\end{equation}
dostaneme metriku ve spojitém tvaru
\begin{equation}
    \label{eq:spojita_gyratonova_metrika}
    \rmd s^2 = \frac{2 \left|\underline{\rmd} \zeta + i \zeta \underline{\rmd}F\right|^2 + 2 \left[i \Theta\left(\zeta H_{,Z} - \bar{\zeta} H_{,\bar{Z}}\right) \underline{\rmd}F - \underline{\rmd}W\right] \rmd U - 2\rmd U \rmd V}{\left[1+\frac{1}{6}\Lambda\left(Z \bar{Z} - UV - U_{+}G\right)\right]^2},
\end{equation}
kde byla zavedena funkce
\begin{equation}
    G \left( Z, \bar{Z}, U \right) = H - Z H_{,Z} - \bar{Z} H_{,\bar{Z}} + W.
\end{equation}

Oproti minulé kapitole se ve funkci $G$ vyskytuje člen $W$. Aplikací transformace \eqref{eq:gyraton_cont_transformation}
na konformní faktor $1 + \frac{1}{6} \Lambda (\eta \bar{\eta} - \matu \matv)$ dostáváme po použití pravidel násobení distribucí v nelineární teorii
\begin{equation}
    1 + \frac{1}{6} \Lambda \left(Z \bar{Z} - UV - U_+ (H - Z H_{,Z} \bar{Z} H_{,\bar{Z}}) - UW \right).
\end{equation}
V případě že $W = 0$ pro $U \leq 0$, platí $UW = U_+ W$. To je ale splněno v \eqref{eq:podminky_F_a_W} a tvar funkce $G$ je tedy ospravedlněn.

Volba $J=0$ dovoluje řešení $F = W = 0$, přičemž se metrika \eqref{eq:spojita_gyratonova_metrika} redukuje na \eqref{eq:nonexp_continuous_metric}.
\eqref{eq:spojita_gyratonova_metrika} je spojitá v případě, že $\underline{\rmd}F$ a $\underline{\rmd}W$ jsou funkce spojité v $U$ a $\underline{\rmd}F$
jde v $U=0$ k nule. V tomto případě je metrika lokálně lipschitzovská a lze využít formalismu Filippových řešení, jako v předchozí kapitole.


\subsection{Frolovův-Fursaevův gyraton}
Dále budeme uvažovat metriku \eqref{eq:nonexp_gyra_distrib_metric_omega} s
\begin{equation}
    J(\eta , \matu) = \frac{\chi}{2 i \eta} \Theta( \matu ),~~~ \Lambda = 0,
\end{equation}
kde $\chi$ je konstantní. Rovnice \eqref{eq:podminky_F_a_W} lze zintegrovat do tvaru
\begin{equation}
    \label{eq:zintegrovane1}
    \begin{split}
        F &= \frac{\chi}{2(Z H_{,Z}-\bar{Z}H_{,\bar{Z}})} \log \frac{Z\bar{Z}+U_{+}\bar{Z}H_{,\bar{Z}}}{Z\bar{Z}+U_{+}ZH_{,Z}}, \\
        W &= \frac{\chi}{2}F,
    \end{split}
\end{equation}
kde uvažujeme hlavní větev logaritmu (aby se zachovala rovnost $\eta = Z$ pro $U \leq 0$).

V případě že $Z H_{,Z} - \bar{Z}H_{,\bar{Z}}=0 \iff \log \frac{Z \bar{Z} + U_{+}\bar{Z}H_{,\bar{Z}}}{Z\bar{Z}+U_{+}ZH_{,Z}}=0$
není tvar \eqref{eq:zintegrovane1} platný. Diferenciální operátor působící na funkci $H$ v levé části ekvivalence
lze (díky $Z = \frac{1}{\sqrt{2}} \rho \exp(i \phi)$) zapsat jako $Z \partial_Z - \bar{Z}\partial_{\bar{Z}} = - i \partial_\phi$,
jedná se tedy o generátor rotace kolem osy $Z = 0$. Tento případ tedy nastává ve všech axiálně symetrických prostoročasech, kde
funkce $H$ závisí pouze na $\rho^2 = 2 Z \bar{Z}$, včetně Aichelburg-Sexlova řešení
$H = b_0 \log(2\eta \bar{\eta}) = b_0 \log (2Z \bar{Z})$, kterým se budeme dále zabývat.
Funkce $F$ a $W$ lze volit ve tvaru
\begin{equation}
    \label{eq:zintegrovane2}
    \begin{split}
        F &= -\frac{\chi}{2} \frac{U_{+}}{Z \bar{Z} + b_0 U_{+}}, \\
        W &= \frac{\chi}{2}F,
    \end{split}
\end{equation}
spojitá metrika nabývá tvaru z třídy Frolovových-Fursaevových gyratonů \cite{Frolov2005} 
\begin{equation}
    \begin{split}
    \rmd s^2 = &2 \left| \rmd Z + U_{+} \left(i \frac{\chi}{2}\frac{\bar{Z}\rmd Z + Z \rmd \bar{Z}}{\bar{Z}\left(Z \bar{Z} + b_0 U_{+}\right)} - b_0 \frac{d\bar{Z}}{\bar{Z}^2}\right) \right|^2 \\
    &- \frac{\chi^2}{2}U_{+}\frac{\bar{Z} \rmd Z \rmd U + Z \rmd \bar{Z} \rmd U}{\left(Z \bar{Z} + b_0 U_{+}\right)^2} - 2 \rmd U \rmd V
    \end{split}
\end{equation}
reprezentujících zobecnění
originální Aichelburg-Sexlovy metriky \cite{Aichelburg_1971} na impulsní vlnu generovanou částicí s
nenulovým vnitřním momentem hybnosti - $\chi \Theta(\matu)$ odpovídá hustotě momentu hybnosti gyratonu \cite{Podolsky2014}.
Při absenci gyratonu ($\chi=0$) se metrika redukuje na standardní Aichelburg-Sexlovo řešení.

Díky tomu, že pracujeme v konformně plochých souřadnicích, nejsou výše uvedené závěry závislé na kosmologické konstantě.
Za funkci $H$ pak můžeme uvažovat například Hottovo-Tanakovo řešení \eqref{eq:Hotta_Tanaka_H_conf_flat_coords}. Pro toto řešení nebudeme explicitně odvozovat
spojitý tvar metriky, později ale využijeme formalismus refrakčních rovnic pro vykreslení geodetického pohybu gyratonického zdroje na (anti-)de Sitterově prostoročasu
s výše uvedenou funkcí $J$ a funkcí $H$ odpovídající právě Hottově-Tanakově řešení. Zmíníme ale, že ekvivalence znemožňující integraci funkcí $F$ a $W$ do tvaru
\eqref{eq:zintegrovane1} neplatí ani v tomto případě, jedná se také o axiálně symetrické řešení.


\section{Refrakční rovnice v impulsních gyratonických prostoročasech}
Stejně jako v případě bez gyratonových členů v předchozí kapitole, i zde k vizualizaci geodetik využijeme refrakčních rovnic,
které byly pro impulsní neexpandující gyratonové prostoročasy s funkcemi $F$ a $W$ ve tvaru
\eqref{eq:zintegrovane1}, resp. \eqref{eq:zintegrovane2}, odvozeny z limity souřadnic \eqref{eq:spojita_gyratonova_metrika}
v $\matu = 0$ v článku \cite{Podolsky_2017}.
Napojovací podmínky prostorových poloh jsou v tomto případě ve tvaru
\begin{equation}
    \label{eq:refrakcni_rovnice_gyra_polohy}
    \begin{split}
        \matu_{i}^{+} &= \matu_{i}^{-}, \\
        \matv_{i}^{+} &= \matv_{i}^{-} + H_{i}, \\
        \eta_{i}^{+} &= \eta_{i}^{-},
    \end{split}
\end{equation}
což odpovídá Penroseovým napojovacím podmínkám \eqref{eq:lepici_podminky},
jejichž tvar nezávisí na přítomnosti gyratonických členů. Co se bude lišit od
výsledků popsaných v minulé kapitole jsou refrakční rovnice pro rychlosti,
\begin{equation}
    \label{eq:refrakcni_rovnice_gyra_rychlosti}
    \begin{split}
        \dot{\matu}_{\rmi}^{+} &= \dot{\matu}_{\rmi}^{-}, \\
        \dot{\matv}_{\rmi}^{+} &= \dot{\matv}_{\rmi}^{-} + H_{\rmi, Z} \dot{\eta}_{\rmi}^{-} + H_{\rmi, \bar{Z}} \dot{\bar{\eta}}_{\rmi}^{-} + \left(H_{\rmi, Z} H_{\rmi, \bar{Z}} - \frac{\chi^2}{4 \eta_{\rmi}^{-} \bar{\eta}_{\rmi}^{-}}\right) \dot{\matu}_{\rmi}^{-} \\
        \dot{\eta}_{\rmi}^{+} &= \dot{\eta}_{\rmi}^{-} + \left(H_{\rmi, \bar{Z}} - \frac{i \chi}{2 \bar{\eta}_{\rmi}^{-}}\right) \dot{\matu}_{\rmi}^{-},
    \end{split} 
\end{equation}
kde gyratonové členy s $\chi$ přispívají ke skoku ve složkách $\dot{\matv}_{\rmi}$ a $\dot{\eta}_{\rmi}$ členy
\begin{equation}
    \Delta \dot{\matv}_{\rmi} = -\frac{\chi^2}{4 \eta_{\rmi}^{-} \bar{\eta}_{\rmi}^{-}} \dot{\matu}_{\rmi}^{-}, ~~~~~~ \Delta \dot{\eta}_{\rmi} = -\frac{i \chi}{2 \bar{\eta}_{\rmi}^{-}} \dot{\matu}_{\rmi}^{-}.
\end{equation}
Tyto skoky navíc se díky $C^1-$regularitě geodetik a spojitosti metriky kompenzují -- normalizace čtyřrychlosti musí být zachována a platí
\begin{equation}
    \Delta \dot{\eta}_{\rmi} \Delta \dot{\bar{\eta}}_{\rmi} = - \dot{\matu}_{\rmi} \Delta \dot{\matv}_{\rmi}.
\end{equation}
Tyto skoky pro $\eta_{\rmi}^{-} \to 0$ rostou nad všechny meze, toto chování ale očekáváme -- v $\eta_{\rmi} = 0$ se nachází bodový zdroj.
Pro $\chi \to 0$ se tyto rovnice redukují na rovnice \eqref{eq:refraction_nonexpanding_velocities} z minulé kapitoly.

Obdobně jako v předchozí kapitole, refrakční rovnice \eqref{eq:refrakcni_rovnice_gyra_rychlosti} můžeme také zapsat v reálných
polárních souřdnicích jako
\begin{equation}
    \begin{split}
    \dot{\matv}_{\rmi}^{+} &= \dot{\matv}_{\rmi}^{-} + \halfsqrt \left(e^{i \varphi_{\rmi}^{-}} H_{\rmi, Z} + e^{-i \varphi_{\rmi}^{-}} H_{\rmi, \bar{Z}}\right) \dot{\rho}_{\rmi}^{-} +
    \frac{i}{\sqrt{2}} \left( e^{i \varphi_{\rmi}^{-}} H_{\rmi, Z} - e^{-i \varphi_{\rmi}^{-}} H_{\rmi, \bar{Z}} \right) \rho_{\rmi}^{-} \dot{\varphi}_{\rmi}^{-} \\
    &~~~+ \left(H_{\rmi, Z} H_{\rmi, \bar{Z}} - \frac{\chi^2}{2(\rho_{\rmi}^{-})^2}\right)\dot{\matu}_{\rmi}^{-}, \\
    \dot{\rho}_{\rmi}^{+} &= \dot{\rho}_{\rmi}^{-} + \halfsqrt \left(e^{-i \varphi_{\rmi}^{-}} H_{\rmi, \bar{Z}} + e^{i \varphi_{\rmi}^{-}} H_{\rmi, Z} \right), \\
    \dot{\varphi}_{\rmi}^{+} &= \dot{\varphi}_{\rmi}^{-} + \left[ \frac{i}{\sqrt{2} \rho_{\rmi}^{-}} \left(e^{i \varphi_{\rmi}^{-}} H_{\rmi, Z} - e^{-i \varphi_{\rmi}^{-}} H_{\rmi, \bar{Z}} \right) - \frac{\chi}{(\rho_{\rmi}^{-})^2}\right] \dot{\matu}_{\rmi}^{-}.
    \end{split}
\end{equation}
Tento tvar potvrzuje interpretaci gyratonových členů jako interního momentu hybnosti částice generující impuls -- 
přítomnost gyraonických členů nemá vliv na radiální složku $\dot{\rho}$, přispívá ale k další změně v
axiální složce $\dot{\varphi_{\rmi}}$ členem $\frac{\chi}{(\rho_{\rmi})^2}$.

\section{Vizualizace geodetik v neexpandujících impulsních gyratonových prostoročasech}
\subsection{Geodetický pohyb v přítomnosti Frolova-Fursaevova gyratonu}

Stejně jako v minulé kapitole využijeme refrakční rovnice \eqref{eq:refrakcni_rovnice_gyra_polohy}, \eqref{eq:refrakcni_rovnice_gyra_rychlosti}
k vizualizaci interakce impulsního Frolova-Fursaevova gyratonu s geodetikami na prostoročasech tvořících pozadí impulsní vlny.
V případě gyratonu je pozadí odlišné, pro $\Lambda = 0$ je prostoročas $\mathcal{M}^{-}$ (před impulsní vlnou) Minkowského prostoročas,
za impulsní vlnou je ale pozadí $\mathcal{M}^+$ popsáno metrikou
\begin{equation}
    \rmd s^2 = - 2 \rmd \matu ~ \rmd \matv + 2 \rmd \eta ~ \rmd \bar{\eta} + i \chi \rmd \matu \left(\frac{\rmd \bar{\eta}}{\bar{\eta}} - \frac{\rmd \eta}{\eta}\right),
\end{equation}
jediné nenulové Christoffelovy symboly jsou
\begin{equation}
    \Gamma^{\matv}_{\eta \eta} = - \frac{i \chi}{2 \eta^2} ~~~~~ \Gamma^{\matv}_{\bar{\eta}\bar{\eta}} = \frac{i \chi}{2 \bar{\eta}^2}.
\end{equation}

Rotující charakter prostoročasu na pozadí za vlnou je také dobře vidět při parametrizaci reálnými souřadnicemi \eqref{eq:complex_coordinates},
kde metrika nabývá tvaru
\begin{equation}
    \label{eq:gyra_plus_uv_metric}
    \rmd s^2 = -2 \rmd \matu ~ \rmd \matv + \rmd x^2 + \rmd y^2 + \frac{2 \chi \rmd \matu (x~\rmd y - y~\rmd x)}{x^2 + y^2}.
\end{equation}
Transformací do standardních polárních souřadnic $x = r \sin{\varphi}, y = r \cos{\varphi}$ přechází metrika \eqref{eq:gyra_plus_uv_metric} do tvaru
\begin{equation}
    \rmd s^2 = \rmd r^2 + r^2 \rmd \varphi^2 - 2 \rmd \matu (\rmd \matv + \chi \rmd \varphi).
\end{equation}
\textcolor{red}{diskuze kauzálních vlastností? při pohybu proti souřadnici $\phi$ se lze pohyovat proti "souřadnicovému" času $\matu + \matv$}

Na následujících vizualizacích je znázorněno chování nejprve nulových a následně časupodobných geodetik pro různé parametry $\chi$, procházející impulsní plochou v
axiálně symetrickém uspořádání kolem gyratonového zdroje na nadploše $\matu = 0, \matv=0$ v různé vzdálenosti $\rho$ od osy symetrie.

\subsubsection{Nulové geodetiky v přítomnosti Frolova-Fursaevova gyratonu}
Na obrázku \ref{fig:gyra_flat_null_b1_chi1_2} jsou pro hodnoty $\chi = \pi, 2\pi$ vizualizovány geodetiky procházející nadplochou $\matu = 0$ s
čtyřrychlostí $\dot{\matu} = 1, \dot{\matv} = 0, \dot{\eta} = 0$, oproti negyratonickému řešení, vizualizovanému v předchozí kapitole, dochází
nejen ke strhnutí geodetik k ose symetrie, ale i k vychýlení vlivem rotačního charakteru prostoročasu. V této vizualizaci se
přítomnost gyratonu projevuje pouze refrakcí na impulsní nadploše -- jediné nenulové Christoffelovy symboly vstupují do geodetické rovnice pro $\matv$.
S rostoucím parametrem $\chi$ je strhávání silnější.
\begin{figure}[ht]
    \centering
    \begin{subfigure}[b]{0.48\textwidth}
        \begin{tikzpicture}
            \node[inner sep=0pt, anchor=south west] (ds) at (0,0)
            {\adjincludegraphics[trim={{.17\width} {.2\height} {.17\width} {.22\height}}, width=1\textwidth, clip]{../img/kap03/flat_gyraton/null/null_gyraton_ring__r_2__mu_1__chi_1pi_uxy.pdf}};
        \end{tikzpicture}
         \caption{$b_0=1, \chi=\pi, \rho=\sqrt{2}$} 
    \end{subfigure}
    \begin{subfigure}[b]{0.48\textwidth}
        \begin{tikzpicture}
            \node[inner sep=0pt, anchor=south west] (ds) at (0,0)
            {\adjincludegraphics[trim={{.17\width} {.2\height} {.17\width} {.22\height}}, width=1\textwidth, clip]{../img/kap03/flat_gyraton/null/null_gyraton_ring__r_3__mu_1__chi_1pi_uxy.pdf}};
        \end{tikzpicture}
         \caption{$b_0=1, \chi=\pi, \rho=\frac{3}{\sqrt{2}}$} 
    \end{subfigure}
    \hfill
    \begin{subfigure}[b]{0.48\textwidth}
        \begin{tikzpicture}
            \node[inner sep=0pt, anchor=south west] (ds) at (0,0)
            {\adjincludegraphics[trim={{.17\width} {.2\height} {.17\width} {.22\height}},width=1\textwidth, clip]{../img/kap03/flat_gyraton/null/null_gyraton_ring__r_2__mu_1__chi_2pi_uxy.pdf}};
        \end{tikzpicture}
         \caption{$b_0=1, \chi=2\pi, \rho=\sqrt{2}$} 
    \end{subfigure}
    \begin{subfigure}[b]{0.48\textwidth}
        \begin{tikzpicture}
            \node[inner sep=0pt, anchor=south west] (ds) at (0,0)
            {\adjincludegraphics[trim={{.17\width} {.2\height} {.17\width} {.22\height}}, width=1\textwidth, clip]{../img/kap03/flat_gyraton/null/null_gyraton_ring__r_3__mu_1__chi_2pi_uxy.pdf}};
        \end{tikzpicture}
         \caption{$b_0=1, \chi=2\pi, \rho=\frac{3}{\sqrt{2}}$} 
    \end{subfigure}
    \caption{Nulové geodetiky procházející impulsní plochou gyratonu.}
    \label{fig:gyra_flat_null_b1_chi1_2}
\end{figure}

Vliv gyratonického charakteru prostoročasu za impulsní plochou je vidět při vizualizaci souřadnice $\matv$ (obrázek \ref{fig:gyra_flat_null_b1_chi1_2_vxy}),
při refrakci ve vizualizovaném případě kladná složka $\dot{\matv}$ přechází v zápornou, vlivem gyratonu je ale vynucováno kladné zrychlení a
geodetika se ve směru $\matv$ obrací.

\begin{figure}[ht]
    \centering
    \begin{subfigure}[b]{0.48\textwidth}
        \begin{tikzpicture}
            \node[inner sep=0pt, anchor=south west] (ds) at (0,0)
            {\adjincludegraphics[trim={{.17\width} {.2\height} {.17\width} {.22\height}}, width=1\textwidth, clip]{../img/kap03/flat_gyraton/null/null_gyraton_ring__r_1__mu_1__chi_2pi_vxy.pdf}};
        \end{tikzpicture}
        \caption{$b_0=1, \chi=2\pi, \rho=1$}
    \end{subfigure}
    \begin{subfigure}[b]{0.48\textwidth}
        \begin{tikzpicture}
            \node[inner sep=0pt, anchor=south west] (ds) at (0,0)
            {\adjincludegraphics[trim={{.17\width} {.2\height} {.17\width} {.22\height}}, width=1\textwidth, clip]{../img/kap03/flat_gyraton/null/null_gyraton_ring__r_2__mu_1__chi_2pi_vxy.pdf}};
        \end{tikzpicture}
        \caption{$b_0=1, \chi=2\pi, \rho=\sqrt{2}$} 
    \end{subfigure}
    \caption{Nulové geodetiky po průchodu impulsní plochou gyratonu.}
    \label{fig:gyra_flat_null_b1_chi1_2_vxy}
\end{figure}

Na obrázku \ref{fig:gyra_skewed_flat_null_b1_chi2} pak vidíme nulové geodetiky procházející impulsní nadplochou
s čtyřrychlostí $\dot{\matu} = 1, \dot{\matv} = 1, \dot{\eta} = 1$, před průchodem $\matu=0$ se v Minkowského části
prostoročasu jednotlivé geodetiky pohybují na nadploše $t, x$.

\begin{figure}[ht]
    \centering
    \begin{subfigure}[b]{1\textwidth}
        \begin{subfigure}[b]{0.48\textwidth}
            \begin{tikzpicture}
                \node[inner sep=0pt, anchor=south west] (ds) at (0,0)
                {\adjincludegraphics[trim={{.14\width} {.15\height} {.15\width} {.22\height}}, width=1\textwidth, clip]{../img/kap03/flat_gyraton/null/null_skewed_gyraton_ring__r_1__mu_1__chi_2pi_uxy.pdf}};
            \end{tikzpicture}
        \end{subfigure}
        \hfill
        \begin{subfigure}[b]{0.48\textwidth}
            \begin{tikzpicture}
                \node[inner sep=0pt, anchor=south west] (ds) at (0,0)
                {\adjincludegraphics[trim={{.12\width} {.1\height} {.15\width} {.18\height}}, width=1\textwidth, clip]{../img/kap03/flat_gyraton/null/null_skewed_top_gyraton_ring__r_1__mu_1__chi_2pi_uxy.pdf}};
            \end{tikzpicture}
        \end{subfigure}
        \caption{$b_0=1, \chi=2\pi, \rho=1$, vykreslené souřadnice $\matu, x, y$} 
    \end{subfigure}
    \hfill
    \begin{subfigure}[b]{1\textwidth}
        \begin{subfigure}[b]{0.48\textwidth}
            \begin{tikzpicture}
                \node[inner sep=0pt, anchor=south west] (ds) at (0,0)
                {\adjincludegraphics[trim={{.14\width} {.15\height} {.15\width} {.22\height}}, width=1\textwidth, clip]{../img/kap03/flat_gyraton/null/null_skewed_gyraton_ring__r_1__mu_1__chi_2pi_vxy.pdf}};
            \end{tikzpicture}
        \end{subfigure}
        \hfill
        \begin{subfigure}[b]{0.48\textwidth}
            \begin{tikzpicture}
                \node[inner sep=0pt, anchor=south west] (ds) at (0,0)
                {\adjincludegraphics[trim={{.12\width} {.10\height} {.15\width} {.18\height}}, width=1\textwidth, clip]{../img/kap03/flat_gyraton/null/null_skewed_top_gyraton_ring__r_1__mu_1__chi_2pi_vxy.pdf}};
            \end{tikzpicture} 
        \end{subfigure}
        \caption{$b_0=1, \chi=2\pi, \rho=1$, vykreslené souřadnice $\matv, x, y$}
    \end{subfigure}
    \caption{Nulové geodetiky procházející impulsní plochou gyratonu, pro zřetelnost vykresleno z různých úhlů pohledu.}
    \label{fig:gyra_skewed_flat_null_b1_chi2}
\end{figure}

\subsubsection{Časupodobné geodetiky v přítomnosti Frolova-Fursaevova gyratonu}
Časupodobné geodetiky byly vyzualizovány v několika uspořádáních, nejprve v případě kdy impulsní
vlnoplochu protínají s tečným vektorem normovaným na jednotku \textcolor{red}{(ono je to vlastně normované na -1, ale to tu asi není podstatné)}
ve směru $\dot{\matu}=\frac{1}{2}, \dot{\matv} = 1, \dot{\eta}=0$, tato vizualizace je zobrazena na
obrázcích \ref{fig:gyra_flat_timelike_b1_chi1_2} a \ref{fig:gyra_flat_timelike_b1_chi2_vxy} a obdobně jako u
nulových geodetik pozorujeme vliv gyratonu strháváním geodetiky ve smyslu rotace.

\begin{figure}
    \centering
    \begin{subfigure}[b]{0.48\textwidth}
        \begin{tikzpicture}
            \node[inner sep=0pt, anchor=south west] (ds) at (0,0)
            {\adjincludegraphics[trim={{.17\width} {.2\height} {.17\width} {.22\height}}, width=1\textwidth, clip]{../img/kap03/flat_gyraton/timelike/gyraton_ring__r_2__mu_1__chi_1pi_uxy.pdf}};
        \end{tikzpicture}
         \caption{$b_0=1, \chi=\pi, \rho=\sqrt{2}$} 
    \end{subfigure}
    \begin{subfigure}[b]{0.48\textwidth}
        \begin{tikzpicture}
            \node[inner sep=0pt, anchor=south west] (ds) at (0,0)
            {\adjincludegraphics[trim={{.17\width} {.2\height} {.17\width} {.22\height}}, width=1\textwidth, clip]{../img/kap03/flat_gyraton/timelike/gyraton_ring__r_3__mu_1__chi_1pi_uxy.pdf}};
        \end{tikzpicture}
         \caption{$b_0=1, \chi=\pi, \rho=\frac{3}{\sqrt{2}}$} 
    \end{subfigure}
    \hfill
    \begin{subfigure}[b]{0.48\textwidth}
        \begin{tikzpicture}
            \node[inner sep=0pt, anchor=south west] (ds) at (0,0)
            {\adjincludegraphics[trim={{.17\width} {.2\height} {.17\width} {.22\height}},width=1\textwidth, clip]{../img/kap03/flat_gyraton/timelike/gyraton_ring__r_2__mu_1__chi_2pi_uxy.pdf}};
        \end{tikzpicture}
         \caption{$b_0=1, \chi=2\pi, \rho=\sqrt{2}$} 
    \end{subfigure}
    \begin{subfigure}[b]{0.48\textwidth}
        \begin{tikzpicture}
            \node[inner sep=0pt, anchor=south west] (ds) at (0,0)
            {\adjincludegraphics[trim={{.17\width} {.2\height} {.17\width} {.22\height}}, width=1\textwidth, clip]{../img/kap03/flat_gyraton/timelike/gyraton_ring__r_3__mu_1__chi_2pi_uxy.pdf}};
        \end{tikzpicture}
         \caption{$b_0=1, \chi=2\pi, \rho=\frac{3}{\sqrt{2}}$} 
    \end{subfigure}
    \caption{Časupodobné geodetiky procházející impulsní plochou gyratonu.}
    \label{fig:gyra_flat_timelike_b1_chi1_2}
\end{figure}

\begin{figure}[ht]
    \centering
    \begin{subfigure}[b]{0.48\textwidth}
        \begin{tikzpicture}
            \node[inner sep=0pt, anchor=south west] (ds) at (0,0)
            {\adjincludegraphics[trim={{.17\width} {.2\height} {.17\width} {.22\height}}, width=1\textwidth, clip]{../img/kap03/flat_gyraton/timelike/gyraton_ring__r_1__mu_1__chi_2pi_vxy.pdf}};
        \end{tikzpicture}
        \caption{$b_0=1, \chi=2\pi, \rho=1$}
    \end{subfigure}
    \begin{subfigure}[b]{0.48\textwidth}
        \begin{tikzpicture}
            \node[inner sep=0pt, anchor=south west] (ds) at (0,0)
            {\adjincludegraphics[trim={{.17\width} {.2\height} {.17\width} {.22\height}}, width=1\textwidth, clip]{../img/kap03/flat_gyraton/timelike/gyraton_ring__r_2__mu_1__chi_2pi_vxy.pdf}};
        \end{tikzpicture}
        \caption{$b_0=1, \chi=2\pi, \rho=\sqrt{2}$} 
    \end{subfigure}
    \caption{Časupodobné geodetiky procházející impulsní plochou gyratonu, částice před impulsem (znázorněné přerušovanou čarou)
    letí ve směru rostoucí souřadnice $\matv$, po impulsu jsou refraktovány zpět a složka čtyřrychlosti $\dot{\matv}$ vlivem refrakce přechází do záporných hodnot.}
    \label{fig:gyra_flat_timelike_b1_chi2_vxy}
\end{figure}

Ve vizualizaci na obrázcích \ref{fig:gyra_flat_timelike_b1_chi1_2_skewed} a \ref{fig:gyra_flat_timelike_b1_chi1_2_vxy_skewed} protínají geodetiky impulsní plochu s tečným vektorem
ve směru $\dot{\matu} = 0,5, \dot{\matv} = 1, \dot{\eta} = 0,5$.

\textcolor{red}{DOPLNIT OBRÁZKY, TYTO JSOU ŠPATNĚ :)}

\begin{figure}
    \centering
    \begin{subfigure}[b]{0.48\textwidth}
        \begin{tikzpicture}
            \node[inner sep=0pt, anchor=south west] (ds) at (0,0)
            {\adjincludegraphics[trim={{.17\width} {.2\height} {.17\width} {.22\height}}, width=1\textwidth, clip]{../img/kap03/flat_gyraton/timelike/skewed_gyraton_ring__r_1__mu_1__chi_1pi_uxy.pdf}};
        \end{tikzpicture}
         \caption{$b_0=1, \chi=\pi, \rho=1$} 
    \end{subfigure}
    \begin{subfigure}[b]{0.48\textwidth}
        \begin{tikzpicture}
            \node[inner sep=0pt, anchor=south west] (ds) at (0,0)
            {\adjincludegraphics[trim={{.17\width} {.2\height} {.17\width} {.22\height}}, width=1\textwidth, clip]{../img/kap03/flat_gyraton/timelike/skewed_gyraton_ring__r_2__mu_1__chi_1pi_uxy.pdf}};
        \end{tikzpicture}
         \caption{$b_0=1, \chi=\pi, \rho=\sqrt{2}$} 
    \end{subfigure}
    \hfill
    \begin{subfigure}[b]{0.48\textwidth}
        \begin{tikzpicture}
            \node[inner sep=0pt, anchor=south west] (ds) at (0,0)
            {\adjincludegraphics[trim={{.17\width} {.2\height} {.17\width} {.22\height}},width=1\textwidth, clip]{../img/kap03/flat_gyraton/timelike/skewed_gyraton_ring__r_1__mu_1__chi_2pi_uxy.pdf}};
        \end{tikzpicture}
         \caption{$b_0=1, \chi=2\pi, \rho=1$} 
    \end{subfigure}
    \begin{subfigure}[b]{0.48\textwidth}
        \begin{tikzpicture}
            \node[inner sep=0pt, anchor=south west] (ds) at (0,0)
            {\adjincludegraphics[trim={{.17\width} {.2\height} {.17\width} {.22\height}}, width=1\textwidth, clip]{../img/kap03/flat_gyraton/timelike/skewed_gyraton_ring__r_2__mu_1__chi_2pi_uxy.pdf}};
        \end{tikzpicture}
         \caption{$b_0=1, \chi=2\pi, \rho=\sqrt{2}$} 
    \end{subfigure}
    \caption{Časupodobné geodetiky procházející impulsní plochou gyratonu.}
    \label{fig:gyra_flat_timelike_b1_chi1_2_skewed}
\end{figure}

\begin{figure}[ht]
    \centering
    \begin{subfigure}[b]{0.48\textwidth}
        \begin{tikzpicture}
            \node[inner sep=0pt, anchor=south west] (ds) at (0,0)
            {\adjincludegraphics[trim={{.17\width} {.2\height} {.17\width} {.22\height}}, width=1\textwidth, clip]{../img/kap03/flat_gyraton/timelike/skewed_gyraton_ring__r_1__mu_1__chi_2pi_vxy.pdf}};
        \end{tikzpicture}
        \caption{$b_0=1, \chi=2\pi, \rho=1$}
    \end{subfigure}
    \begin{subfigure}[b]{0.48\textwidth}
        \begin{tikzpicture}
            \node[inner sep=0pt, anchor=south west] (ds) at (0,0)
            {\adjincludegraphics[trim={{.17\width} {.2\height} {.17\width} {.22\height}}, width=1\textwidth, clip]{../img/kap03/flat_gyraton/timelike/skewed_gyraton_ring__r_2__mu_1__chi_2pi_vxy.pdf}};
        \end{tikzpicture}
        \caption{$b_0=1, \chi=2\pi, \rho=\sqrt{2}$} 
    \end{subfigure}
    \caption{Časupodobné geodetiky procházející impulsní plochou gyratonu.}
    \label{fig:gyra_flat_timelike_b1_chi1_2_vxy_skewed}
\end{figure}