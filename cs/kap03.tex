\chapter{Neexpandující gravitační vlny s gyratonovými členy}
Distribuční vyjádření metriky impulsní vlny \eqref{eq:nonexp_distr_metric_omega} není nejobecnější metrikou popisující
impulsní vlny. Už Brinkmann \textcolor{red}{(nezapomenout Brinkmanna v úvodu BP!)} uvažoval i tzv. mimodiagonální členy, se kterými pak (i v případě nenulové kosmologické konstanty)
metrika nabývá tvaru
\begin{equation}
    \label{eq:nonexp_gyra_distrib_metric_omega}
    \begin{split}
        \mathrm{d}s^2=&\frac{2\mathrm{d}\eta~\mathrm{d}\bar{\eta} - 2 \mathrm{d}\mathcal{U}~\mathrm{d}\mathcal{V} + 2H(\eta, \bar{\eta}) \delta(\mathcal{U}) 
        ~\mathrm{d}\mathcal{U}^2}{\left[1+\frac{1}{6}\Lambda(\eta \bar{\eta}-\mathcal{U}\mathcal{V})\right]^2} \\
        &+ \frac{2J\left(\eta, \bar{\eta}, \mathcal{U}\right) \mathrm{d}\eta~\mathrm{d}\mathcal{U}
        +2\overline{J}\left(\eta, \bar{\eta}, \mathcal{U}\right) \mathrm{d}\bar{\eta}~\mathrm{d}\mathcal{U}}{\left[1+\frac{1}{6}\Lambda(\eta \bar{\eta}-\mathcal{U}\mathcal{V})\right]^2}.
    \end{split}
\end{equation}
Obvyklým postupem je odstranění členů s funkcí $J$ vhodnou souřadnicovou transformací, to ale vede k odstranění
možného rotačního charakteru zdroje gravitační vlny, taková transformace pak není globální, dochází k
zanedbání topologických vlastností celého prostoročasu.

\section{Zobecnění impulsních vln na prostoročasy s gyratonovými členy}
Konstrukce neexpandujících gracitačních vln s gyratonovými členy je obdobná případu bez gyratonových
členů. Opět můžeme využít Penroseovu "cut and paste"\ metodu, v případě gyratonických prostoročasů ale bude
konstrukce totožná s konstrukcí popsanou v \eqref{sec:cut_and_paste_konstrukce1} a jak je ukázáno v článku \cite{Podolsky_2017}, Penroseovy lepící podmínky 
v přítomnosti gyratonů nabývají tvaru \eqref{eq:lepici_podminky}. \textcolor{red}{(přeformulovat něják lépe...)},

\subsection{Zobecnění spojitého tvaru metriky}
Tvar spojité metriky pro gyratonové neexpandující impulsní vlny obdržíme zobecněním transformace \eqref{eq:nonexp_cont_full_transform},
do tvaru který nalezli Podolský, Švarc, Säman a Steinbauer v \cite{Podolsky_2017}
\begin{equation}
    \label{eq:gyraton_cont_transformation}
    \begin{split}
        \matu &= U \\
        \matv &= V + \Theta H + U_{+} H_{,Z} H_{,\bar{Z}} + W \\
        \eta &= \left(Z + U_{+} H_{,\bar{Z}}\right) \exp \left(i F \right),
    \end{split}
\end{equation}

kde opět $H = H(Z, \bar{Z})$, zároveň funkce $W = W(Z, \bar{Z}, U)$ a $F = F(Z, \bar{Z}, U)$ jsou reálné a splňují
\begin{equation}
    \label{eq:podminky_F_a_W}
    \begin{split}
        F_{,U} &= \frac{i\bar{J}}{Z + U_{+}H_{,\bar{Z}}} \exp{\left(-iF\right)}, \\
        W_{,U} &= -J \bar{J}, \\
        W &= 0 \text{ pro } U \leq 0.
    \end{split}
\end{equation}
Zavedením
\begin{equation}
    \zeta \equiv Z + U_{+} H_{,\bar{Z}}
\end{equation}
a prostorového diferenciálu $\underline{\rmd}$ tak, že
\begin{equation}
    \label{eq:spatial_differential_on_functions}
    \begin{split}
        \underline{\rmd}\zeta &\equiv \rmd Z + U_{+}(H_{,\bar{Z}Z}\rmd Z + H_{,\bar{Z}\bar{Z}}\rmd \bar{Z}), \\
        \underline{\rmd}H &\equiv H_{,Z}\rmd Z + H_{,\bar{Z}}\rmd \bar{Z}, \\
        \underline{\rmd}F &\equiv F_{,Z}\rmd Z + F_{,\bar{Z}}\rmd \bar{Z}, \\
        \underline{\rmd}W &\equiv W_{,Z}\rmd Z + W_{,\bar{Z}}\rmd \bar{Z}.
    \end{split}
\end{equation}
Dosazením \eqref{eq:gyraton_cont_transformation} do \eqref{eq:nonexp_gyra_distrib_metric_omega} a využitím vztahů \eqref{eq:podminky_F_a_W},
\eqref{eq:spatial_differential_on_functions} a multiplikativních pravidel z nelineární teorie distribucí,
\begin{equation}
    \Theta^2 = \Theta, ~~~~~~ \Theta U_{+} = U_{+},
\end{equation}
dostaneme metriku ve spojitém tvaru
\begin{equation}
    \label{eq:spojita_gyratonova_metrika}
    \rmd s^2 = \frac{2 \left|\underline{\rmd} \zeta + i \zeta \underline{\rmd}F\right|^2 + 2 \left[i \Theta\left(\zeta H_{,Z} - \bar{\zeta} H_{,\bar{Z}}\right) \underline{\rmd}F - \underline{\rmd}W\right] \rmd U - 2\rmd U \rmd V}{\left[1+\frac{1}{6}\Lambda\left(Z \bar{Z} - UV - U_{+}G\right)\right]^2}.
\end{equation}

Volba $J=0$ dovoluje řešení $F = W = 0$, přičemž se metrika \eqref{eq:spojita_gyratonova_metrika} redukuje na \eqref{eq:nonexp_continuous_metric}.
\eqref{eq:spojita_gyratonova_metrika} je spojitá v případě, že $\underline{\rmd}F$ a $\underline{\rmd}W$ jsou funkce spojité v $U$ a $\underline{\rmd}F$
jde v $U=0$ k nule. V tomto případě je metrika lokálně lipschitzovská a lze využít formalismu Filippových řešení, jako v předchozí kapitole.

\subsection{Gyraton s $J \propto \Theta$}



\section{Interakce gyratonů s testovacími částicemi}
\subsection{Refrakční rovnice}
\textcolor{red}{Stejná struktura jako předchozí kapitola, stačí přepsat a trochu okomentovat
(např. "delty" ve skocích v čtyřrychlosti - započítat metrické členy s J)}