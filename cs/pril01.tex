\section{Pythonový balíček GRImpulsiveWaves}
Pythonový Balíček GRImpulsiveWaves vznikl k vizualizaci geodetik procházejících impulsní plochou zkoumaných gravitačních vln.
Usnadňuje vykreslování modelových situací v předdefinových prostoročasech obsahujících impulsní vlnu pomocí
odvozených refrakčních rovnic a numerické integrace rovnice geodetiky. Jak ale ukáží příklady použití dále, lze
jej poměrně jednoduše rozšířit i na vizualizaci modelových prostoročasů bez přítomnosti impulsních vln.
Balíček také obsahuje modul pro zkoumání geodetického chaosu pomocí numerického výpočtu chaotických indikátorů
\textcolor{red}{(to bych ještě rád doprogramoval, není to příliš těžké)} FmLCE, FLI a MEGNO \textcolor{red}{(možná jen některé z nich,
protože ale jde jenom o porovnávání vektoru geodetické deviace tak implementace všech nebude problémová)}

\subsection{Instalace balíčku}
\textcolor{red}{Nejprve hodit balíček na pip}

\subsection{Vizualizace geodetik v prostoročasech s impulsní vlnou}
GRImpulsiveWaves reprezentuje profil impulsní vlny pomocí vlastní třídy dědící od třídy \verb|RefractionSolution|.
Třída \verb|RefractionSolution| má implementovanou obecnou metodu pro numerickou integraci geodetické rovnice\\
\textcolor{red}{Stručně o tom jak to funguje, hlavně příklady (+ jak správně psát kód v TeXu)}

