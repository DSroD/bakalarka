%%% Šablona pro jednoduchý soubor formátu PDF/A, jako treba samostatný abstrakt práce.

\documentclass[12pt]{report}

\usepackage[a4paper, hmargin=1in, vmargin=1in]{geometry}
\usepackage[a-2u]{pdfx}
\usepackage[czech]{babel}
\usepackage[utf8]{inputenc}
\usepackage[T1]{fontenc}
\usepackage{lmodern}
\usepackage{textcomp}

\begin{document}

%% Nezapomeňte upravit abstrakt.xmpdata.

V hlavní části této práce jsou rekapitulovány metody konstrukce prostoročasů
s neexpandujícími impulzními gravitačními vlnami, přesněji prostoročasů z limitního případu Kundtovy třídy zahrnující
prostoročasy s gyratony. Dále užitím metody $\mathcal{C}^1$-matchingu, která vede na takzvané refrakční rovnice
udávající podmínky napojení geodetického pohybu skrze impulz, zkoumáme chování volných testovacích částic.
Je provedena fyzikální analýza a vizualizace geodetického
pohybu pro vybrané případy studovaných prostoročasů.
V rámci této práce byl vytvořen balíček \emph{GRImpulsiveWaves} pro programovací jazyk Python sloužící k
interaktivní vizualizaci geodetického pohybu v impuzních prostoročasech právě na základě refrakčních rovnic.
\end{document}
